%%%%%%%%%%%%%%%%%%%%%%%%%%%%%%%%%%%%%%%%%%%%%%%%%%%%%%%
% % 
% % 		Project 0.10 	Factoring and Primality Testing
%%
% %	 Math 371: 		Number Theory, Spring 2016
% % 	Author(s): 		Jodie Miu, Margaret Dorsey
% % 	Last Updated: 	04 / 14 / 2016	(see gitlog.txt for revision history)
% % 
%%%%%%%%%%%%%%%%%%%%%%%%%%%%%%%%%%%%%%%%%%%%%%%%%%%%%%%
% % Title and author(s)
%%%%%%%%%%%%%%%%%%%%%%%%%%%%%%%%%%%%%%%%%%%%%%%%%%%%%%%

\title{ Factoring and Primality Testing \thanks {Math 371 Number Theory, Spring 2015, Dr. Anurag Agarwal}}
\author{	Jodie Miu 
        			\and
       		 Margaret Dorsey
	}
	
	
%%%%%%%%%%%%%%%%%%%%%%%%%%%%%%%%%%%%%%%%%%%%%%%%%%%%%%%
%%       
%%	You use the following command
 %%                 \thanks{ Message here}
 %%   for acknowledgements, affiliations, and whatnot after your name. I don't know if we want to acknowledge the class?
 %%		-Margaret
 %% 
%%%%%%%%%%%%%%%%%%%%%%%%%%%%%%%%%%%%%%%%%%%%%%%%%%%%%%%

\documentclass{article}

%%%%%%%%%%%%%%%%%%%%%%%%%%%%%%%%%%%%%%%%%%%%%%%%%%%%%%%
% %
% % The graphicx package allows importing of 
% % epsf or .eps files for vector graphics.
% %
%%%%%%%%%%%%%%%%%%%%%%%%%%%%%%%%%%%%%%%%%%%%%%%%%%%%%%%
\usepackage{graphicx}
%%%%%%%%%%%%%%%%%%%%%%%%%%%%%%%%%%%%%%%%%%%%%%%%%%%%%%%
% %
% % I adapt the already existing abstract environment for our Introduction, because we have no abstract.
%%		-Margaret
% %
%%%%%%%%%%%%%%%%%%%%%%%%%%%%%%%%%%%%%%%%%%%%%%%%%%%%%%%
\newenvironment{intro}
  {\renewcommand\abstractname{Introduction}\begin{abstract}}
  {\end{abstract}}

%%%%%%%%%%%%%%%%%%%%%%%%%%%%%%%%%%%%%%%%%%%%%%%%%%%%%%%
%%
%%	We can change the formatting, fonts, size, etc at any time up here if necessary.
%%		-Margaret
%%
%%%%%%%%%%%%%%%%%%%%%%%%%%%%%%%%%%%%%%%%%%%%%%%%%%%%%%%


%%%%%%%%%%%%%%%%%%%%%%%%%%%%%%%%%%%%%%%%%%%%%%%%%%%%%%%
% % There are a lot of commands for theorems, declarations, etc.
% %
% % Here are some common ones. 
%%		-Margaret
%%%%%%%%%%%%%%%%%%%%%%%%%%%%%%%%%%%%%%%%%%%%%%%%%%%%%%%
%%
%%\newtheorem{theorem}{Theorem}
%%\newtheorem{lemma}{Lemma}
%%\newtheorem{proposition}{Proposition}
%%\newtheorem{scolium}{Scolium}   %% Maybe common was an overstatement..
%%\newtheorem{definition}{Definition}
%%\newenvironment{proof}{{\sc Proof:}}{~\hfill QED}
%%\newenvironment{AMS}{}{}
%%\newenvironment{keywords}{}{}
%%
%%%%%%%%%%%%%%%%%%%%%%%%%%%%%%%%%%%%%%%%%%%%%%%%%%%%%%%

\begin{document}
\newpage
\maketitle

%%%%%%%%%%%%%%%%%%%%%%%%%%%%%%%%%%%%%%%%%%%%%%%%%%%%%%% 
%%
%%	I use the environment I made above here for our introduction.
%%		-Margaret
%%
%%%%%%%%%%%%%%%%%%%%%%%%%%%%%%%%%%%%%%%%%%%%%%%%%%%%%%%


\begin{intro}

    %% Introduction goes here
    
\end{intro}

%%%%%%%%%%%%%%%%%%%%%%%%%%%%%%%%%%%%%%%%%%%%
%%
%% 	We won't use these most likely, but just so we can add them easily if needed
%%		-Margaret
%%
%% \begin{keywords}
%%  \LaTeX, typesetting
%%  \end{keywords}

%%  \begin{AMS}
%%   50C60, 18C25
%%  \end{AMS}

%%%%%%%%%%%%%%%%%%%%%%%%%%

%%%%%%%%%%%%%%%%%%%%%%
%%	The paper begins...
%%%%%%%%%%%%%%%%%%%%%%

%%%%%%%%%%%%%%%%%%%%%%%%
%%
%%  \section{ sectionName }
%%	\subsection{ name}
%%	\subsubsection{name}
%%	etc..
%%  Use sectioning commands for headings. Often longer articles are divided into a few sections.
%%
%%%%%%%%%%%%%%%%%%%%%%%%%%%%%%%%
%%
%%  use /par to start a new paragraph, // for a simple line break.
%%
%%  You must use labelling commands, e.g. \label, \ref,
%%  \bibitem, and \cite to refer to sections of your document, such as
%%  see Section~\ref{sectionname}, see Figure~\ref{figurename}, or bibliography entries,
%%  such as see~\cite{citationname}.  Otherwise the look of the numbers, and sometimes the numbers
%%  themselves, will get messed up.
%%
%%%%%%%%%%%%%%%%%%%%%%%%%%%%%%%%%

\section{History}\label{History}
  
%% Writing goes here

\section{Background}\label{Background}

\section{Results}\label{Results}

\section{Applications}\label{Applications}

\section{Conclusion}\label{Conclusion}



\begin{thebibliography}{9}

%%
%% Filled these in, pretty sure the formatting is not correct for academia. Will fix when we can figure that out
%%
         
   \bibitem{ Crandall }
         {\sc Crandall, Richard \and Pomerance, Carl,}
         {\em Prime Numbers: A Computational Perspective, $2^{nd}$ ed.}
         Springer 2005.
         
   \bibitem{ Koblitz }
         {\sc Koblitz, Neal,}
         {\em A Course in Number Theory and Cryptography, $2^{nd}$ ed.}
         Springer 1994.
    
    \bibitem{ Mollin }
         {\sc Mollin, Richard,}
         {\em A Brief History of Factoring and Primality Testing BC (Before Computers)}
         Mathematics Magazine, Vol. 75 No. 1 (Feb 2002).
         
    \bibitem{ Riesel }
         {\sc Riesel, Hans,}
         {\em Prime Numbers and Computer Methods for Factorization, $2^{nd}$ ed.}
         Birkhauser Boston 1994.

%%
%% Citations go here
%%

\end{thebibliography}

%%
%% The following sections are something we should fill in together at some point,
%% They are not important right now but are nice to have, I'm not really sure what is relevant to put there.
%%	-Margaret
%%

\section*{About the author:}


%%
%%	Stuff goes here
%%

\subsection*{Margaret Dorsey}
   Computational Mathematics, Game Design and Development, Computer Science
   med7068@rit.edu

\subsection*{Jodie Miu}
   Computer Science
   jm7481@rit.edu 
   
   
%%%%%%%%%%%%%%%%%%%%%%%%%%%%%%%%%%
%% 
%%	Some notes and explanations
%%
%%%%%%%%%%%%%%%%%%%%%%%%%%%%%%%%%%	

%%%%%%%%%%%%%%
%%
%% Matrix Notation - don't know if we'll end up using it, but it is kind of strange so here's an example
%%
%%%%%%%%%%%%%%


%% $$
%%  \left[
%% 	\begin{array}{rrrrrrrr}
 %% 		0 & -1 & 0 & 1 & 0 & 0 & 0 & 0  \\
 %% 		1 & 0 & 0 & 0 & -1 & 0 & 0 & 0  \\
 %% 		0 & 0 & \alpha & \alpha & -\alpha & -\alpha & 0 & 0  \\
 %% 		\alpha & \alpha & 0 & 0 & 0 & 0 & -\alpha & -\alpha  \\
 %% 		0 & 0 & 1 & 0 & 0 & 0 & -1 & 0  \\
 %% 		0 & 0 & 0 & 0 & 0 & 1 & 0 & -1
 %% 	\end{array}
 %% \right]
 %%	$$
 
%%%%%%%%%%%%%%
%%
%% Figures - this is how you use a figure made from Mathematica or Maple in .eps format. The .eps file must be in the same directory
%% .eps file must be pure ascii mode, no preview, no thumbnail
%% I have both maple and mathematica if we need figures, so don't bother to go and buy a license.
%%		-Margaret
%%
%%%%%%%%%%%%%%

%%	\begin{figure}[htb]
%%     \centering
%%     \includegraphics[width=2.5in]{para01.eps}
%%     \caption{$y = x^{2}$\label{ParabFig}}
 %%	 \end{figure}

\end{document}
