%%%%%%%%%%%%%%%%%%%%%%%%%%%%%%%%%%%%%%%%%%%%%%%%%%%%%%%
% % 
% % 		Project 0.10 	Factoring and Primality Testing
%%
% %	 Math 371: 		Number Theory, Spring 2016
% % 	Author(s): 		Jodie Miu, Margaret Dorsey
% % 	Last Updated: 	05 / 14 / 2016	(see gitlog.txt for revision history)
% % 
%%%%%%%%%%%%%%%%%%%%%%%%%%%%%%%%%%%%%%%%%%%%%%%%%%%%%%%%

%%%%%%%%%%%%%%%%%%%%%%%%%%%%%%%%%%%%%%%%%%%%%%%%%%%%%%%
% % Title and author(s)
%%%%%%%%%%%%%%%%%%%%%%%%%%%%%%%%%%%%%%%%%%%%%%%%%%%%%%%

\title{ Factoring and Primality Testing}
\author{	Jodie Miu 
        			\and
       		 Margaret Dorsey
	}
\date{Spring 2016}

\documentclass{article}

%%%%%%%%%%%%%%%%%%%%%%%%%%%%%%%%%%%%%%%%%%%%%%%%%%%%%%%
% %
% % The graphicx package allows importing of 
% % epsf or .eps files for vector graphics.
%%
%% hyphenat package is used to prevent hyphenation and force line breaks
%%
%% biblatex is for our references section,  filecontents is so I can cheat and not use a .bib file for biblatex
%%
%% titlesec is for making the section headings larger
%%
%% The other packages are primarily for formatting and display of math, theorems, and definitions.
%%		-Margaret
%%%%%%%%%%%%%%%%%%%%%%%%%%%%%%%%%%%%%%%%%%%%%%%%%%%%%%%
\usepackage{graphicx}
\usepackage{amsthm}
\usepackage{amsfonts}
\usepackage{amssymb}
\usepackage{amsmath}
\usepackage[none]{hyphenat} 
\usepackage{comment}
\usepackage[numbers]{natbib}
\usepackage{titlesec}
\usepackage{filecontents}

%%%%%%%%%%%%%%%%%%%%%%%%%%%%%%%%%%%%%%%%%%%%%%%%%%%%%%
%%
%% Constructing and including the bibliography/references section
%%
%%%%%%%%%%%%%%%%%%%%%%%%%%%%%%%%%%%%%%%%%%%%%%%%%%%%%%
\begin{filecontents}{references.bib}
@book{riesel,
title = {{P}rime {N}umbers and {C}omputer {M}ethods for {F}actorization},
edition = {2},
author = {Riesel, Hans},
publisher = {Birkhauser Boston},
year = {1994},
isbn = {0-8176-3743-5}
}

@book{crandall,
title = {{P}rime {N}umbers: {A} {C}omputational {P}erspective},
edition = {2},
author = {Crandall, Richard and Pomerance, Carl},
publisher = {Springer},
year = {2005},
isbn = {0-387-25282-7}
}

@book{koblitz,
title = {{A} {C}ourse in {N}umber {T}heory and {C}ryptography},
edition = {2},
author = {Koblitz, Neal},
publisher = {Springer},
year = {1994},
isbn = { 0-387-94293-9}
}

@book{giblin,
title = {{P}rimes and {P}rogramming: {A}n {I}ntroduction to {N}umber {T}heory with {C}omputing},
edition = {2},
author = {Giblin, Peter},
publisher = {Cambridge University Press},
year = {1993},
isbn = {0-521-40182-8}

@article{historyref,
author =       {Richard A. Mollin},
title =        {{A} {B}rief {H}istory of {F}actoring and {P}rimality {T}esting {B}efore {C}omputers},
 journal =      {Mathematics Magazine},
 volume =       {75},
 number =       {1},
 pages =        {18-29},
 year =         {2002},
 url = 		 {http://www.jstor.org/stable/3219180}
}

}
\end{filecontents}


%%%%%%%%%%%%%%%%%%%%%%%%%%%%%%%%%%%%%%%%%%%%%%%%%%%%%%%
% %
% % I adapt the already existing abstract environment for our Introduction, because we have no abstract.
%%		-Margaret
% %
%%%%%%%%%%%%%%%%%%%%%%%%%%%%%%%%%%%%%%%%%%%%%%%%%%%%%%%
\newenvironment{intro}
  {\renewcommand\abstractname{Introduction}\begin{abstract}}
  {\end{abstract}}

%%%%%%%%%%%%%%%%%%%%%%%%%%%%%%%%%%%%%%%%%%%%%%%%%%%%%%%
%%
%%	We can change the formatting, fonts, size, etc at any time up here if necessary.
%%		-Margaret
%%
%%%%%%%%%%%%%%%%%%%%%%%%%%%%%%%%%%%%%%%%%%%%%%%%%%%%%%%
\titleformat*{\section}{\LARGE\bfseries}
\titleformat*{\subsection}{\Large\bfseries}
\titleformat*{\subsubsection}{\large\bfseries}


%%%%%%%%%%%%%%%%%%%%%%%%%%%%%%%%%%%%%%%%%%%%%%%%%%%%%%%
% % There are a lot of commands for theorems, declarations, etc.
% %
% % Here are some common ones. Use a * after newtheorem to disable numbering.
%%		-Margaret
%%%%%%%%%%%%%%%%%%%%%%%%%%%%%%%%%%%%%%%%%%%%%%%%%%%%%%%
%%
\newtheorem*{theorem}{Theorem}
\newtheorem*{lemma}{Lemma}
%%\newtheorem*{proposition}{Proposition}
%%\newtheorem*{scolium}{Scolium}   %% Maybe common was an overstatement.. - Margaret
\newtheorem*{definition}{Definition}
%%\newenvironment{proof}{{\sc Proof:}}{~\hfill QED}
%%
%%%%%%%%%%%%%%%%%%%%%%%%%%%%%%%%%%%%%%%%%%%%%%%%%%%%%%%

\begin{document}
\newpage
\maketitle

%%%%%%%%%%%%%%%%%%%%%%%%%%%%%%%%%%%%%%%%%%%%%%%%%%%%%%% 
%%
%%	I use the environment I made above here for our introduction.
%%		-Margaret
%%
%%%%%%%%%%%%%%%%%%%%%%%%%%%%%%%%%%%%%%%%%%%%%%%%%%%%%%%


\begin{intro}

    \par While the identification of primes and the factorization are clearly motivated by the computational needs of cryptography in modern times, the roots of current methods can be traced back as far as Eratosthenes' sieve, developed in the $3^{rd}$ century BCE. While it is difficult to say why mathematicians continued to study factorization, save for hunger for knowledge, the findings in these areas contributed immensely to other areas of both Number Theory and other fields of mathematics, particularly in allowing for work with computation using large numbers that would otherwise have been unfeasible in a pre-computing world.
	\par This paper attempts to illustrate the progression from the very early, intuitive results and methods of Eratosthenes, Fermat, Euler, and Lehmer, among others, to the more sophisticated modern methods more commonly used today, while proving some of the more approachable results. The intention of the author is to build the theory of primality testing and factorization from elementary knowledge, so that a reader with a modest mathematical background may understand each method as it is introduced, with each building upon previous methods in a way that seems natural and intuitive.
	\par Given the computational nature of the topic, some comments will be made on the computational viability of methods where appropriate. 
\end{intro}

%%%%%%%%%%%%%%%%%%%%%%
%%	The paper begins...
%%%%%%%%%%%%%%%%%%%%%%

%%%%%%%%%%%%%%%%%%%%%%%%
%%
%%  \section*{ sectionName }
%%	\subsection*{ name}
%%	\subsubsection*{name}
%%	etc..
%%  Use sectioning commands for headings. Often longer articles are divided into a few sections.
%%
%%%%%%%%%%%%%%%%%%%%%%%%%%%%%%%%
%%
%%  use /par to start a new paragraph, // for a simple line break.
%%
%%  To reference using biblatex, use \cite{name}
%%  -Margaret
%% 
%%
%%%%%%%%%%%%%%%%%%%%%%%%%%%%%%%%%

\section*{History}\label{History}
  
%% Writing goes here

\section*{Background}\label{Background}

In this section, we examine a few topics that will be necessary in the main paper of which the reader may be unaware. 


\subsection*{Primitive Roots}
\begin{definition}
A number $g$ is a \textit{primitive root} modulo $m$ if every number $a$ coprime to $n$ is congruent to some power of $g$ modulo $n$. This $k$ is called the index of $a$ to the base $g$ modulo $n$.
\end{definition}
\par The primitive residue classes modulo $n$ have many useful properties, but we will not examine them very thoroughly - we only state without proof that $n$ has a primitive root if it is of the form $2, 4, p^k, \text{ or } 2p^k$ where $p$ is an odd prime and $k \geq 1$, and if an integer $n$ has a primitive root, then it has $\phi(\phi(n))$ of them. Additionally, the lowest power of a primitive root $a$ modulo $n$ that is equivalent to $1$ modulo $n$ is $\phi(n)$.

We conclude with a small lemma that will be useful to us.
\begin{lemma}
If $k$ is the least positive integer such that $a^k \equiv 1 \pmod n$, $a^j \equiv 1 \pmod n$, and $k < j$, then $k \mid j$.
\end{lemma}
\begin{proof}
Let $k$ be the least power such that $a^k \equiv 1 \pmod n$. $j = k + m$, with $m \in \mathbb N$. if $k \nmid m$, then $m = qk+ r$, $ 0 < r < m$, and $j = (q+1)k + r$. However, $a^j = a^{(q+1)k}a^r \equiv a^r \equiv 1 \pmod n$ contradicts the minimality of $k$. Thus $k$ must divide $j$.
\end{proof}
\subsection*{Quadratic Residues}
\newtheorem*{quadraticresiduedefinition}{Quadratic Residue}
\begin{quadraticresiduedefinition}
Let $a$ and $n$ be integers such that $(a,n) = 1$. If the congruence
	$$x^2 \equiv a \pmod n$$
has solutions $x$, then $a$ is a \textit{quadratic residue} of $n$. If there are no such solutions, then $a$ is a \textit{quadratic non-residue} modulo $n$.
\end{quadraticresiduedefinition}

\par It should be noted that the condition $(a,n) = 1$ allows us to consider only the so-called \textit{primitive} residue classes modulo $n$, or those classes that are relatively prime to $n$ when searching for quadratic residues. 

\par We now have a result related to quadratic residues.
 \begin{theorem}
 The product of two quadratic residues $a$ and $b$ modulo $n$ is always a quadratic residue of $n$, and the product of two quadratic non-residues $\alpha$ and $\beta$ modulo $n$ is always a quadratic non-residue.
 \end{theorem}
 \begin{proof}
 Suppose that two integers $a$ and $a$ both relatively prime to $n$ are quadratic residues modulo $n$. Thus $x^2 \equiv a
  \pmod n$ and $y^2 \equiv b \pmod n$ for some $x$ and $y$. Because $a$ and $b$ are relatively prime to $n$, we can say
   $x^2y^2 = (xy)^2 \equiv ab \pmod n$ , which gives us our first result.
 \par Now suppose there is an $x$ such that  $x^2 \equiv \alpha \pmod n$, but no $y$ such that  $y^2 \equiv \beta \pmod n$,
  and there exists some $z$ such that $z^2 \equiv \alpha\beta \pmod n$. This gives us $z^2 \equiv x^2 \beta \pmod n$ , and thus 
  $\left( \frac{z}{x} \right)^2 \equiv \beta \pmod n$, which contradicts our assumption that $\beta$ is a non-quadratic residue.
 \end{proof}
 
 \par Note that we do not address the product of two non-residues - we will state without proof that for primes, the product of two non-residues is a residue, and that the issue is more complex for composite numbers. 
 
 
\subsubsection*{Legendre's Symbol, Euler's Criterion, and Jacobi's Symbol}
We define Legendre's Symbol $\left( \frac{a}{p} \right)$ as a symbol given the value $1$ if $a$ is a quadratic residue of $p$ and the value $-1$ if $a$ is a quadratic non-residue of $p$, where $p$ is a prime.
\par It is, of course, possible to compute Legendre's Symbol directly by trying every congruence class of $p$, but luckily there is a more elegant way, given by Euler:

\newtheorem*{eulercriterion}{Euler's Criterion}
\begin{eulercriterion}
If $(a,p)  = 1$ and $p$ is an odd prime, then 
	$$ \left( \frac{a}{p} \right)  \equiv a^{\frac{p-1}{2}} \pmod p$$
\end{eulercriterion}
\begin{proof}
The result can be broken into two results:
	$$a^{\frac{p-1}{2}} \equiv 1 \pmod p \text{ if and only if } a \text{ is a quadratic residue of } p$$
	$$\text{ and }$$
	$$a^{\frac{p-1}{2}} \equiv -1 \pmod p \text{ if and only if } a \text{ is a quadratic non-residue of } p$$
	
\par We begin with the second of these. Let $a$ be a quadratic non-residue of a prime $p$, and $b$ be some natural number less than $p$. $bx \equiv a$ has a unique solution $x = b^{-1}$, because $(b,p) = 1$. However, $b \not \equiv b^{-1} \pmod p$, because otherwise $b^2 \equiv a \pmod p$ contradicts the assumption that $a$ is a quadratic non-residue. Thus all natural numbers less than $p$ can be paired into $\frac{p-1}{2}$ pairs $(m,n)$ such that $mn \equiv a \pmod p$. 
\par Multiplying these pairs together, we have a product of  $\frac{p-1}{2}$ integers all equivalent to $a$ modulo $p$, which will be equivalent to $a^\frac{p-1}{2}$ modulo $p$ and  equal to $(p-1)!$. By Wilson's Theorem, we also have $(p-1)! \equiv -1 \pmod p$, and so when $a$ is a quadratic non-residue of $p$, $a^\frac{p-1}{2} \equiv -1 \pmod p$.
\par Now let $a$ be such that $a^\frac{p-1}{2} \equiv -1 \pmod p$. By Wilson's Theorem, $a^\frac{p-1}{2} \equiv (p-1)! \pmod p$, and thus this implies that the product of all natural  numbers $b < p$ produce $\frac{p-1}{2}$ factors equivalent to $a$. If any $b$ exists such that $bx \equiv a \pmod p$ has its unique solution $x$ equivalent to $b$, then $b$ will not be able to form a product equivalent to $a$ when paired with any other factor of $(p-1)!$, which is a contradiction. Thus there is no $b$ such that $b^2 \equiv a \pmod p$, and $a$ is a quadratic non-residue.
\par Now let $a$ be a quadratic residue of $p$. We can pick a natural number $b < p$ such that $b^2 \equiv a \pmod p$, and thus $ b^{p-1} \equiv a^\frac{p-1}{2} \pmod p$. By Fermat's Theorem, we have $ a^\frac{p-1}{2} \equiv 1 \pmod p$.
\par Finally, we examine the case that  $a^\frac{p-1}{2} \equiv 1 \pmod p$ for some arbitrary $a$ relatively prime to $p$. Let $b$ be a primitive root modulo $p$ such that $a$ can be written as $b^j$ for some $j$. From this we have $b^{j \cdot \frac{p-1}{2}} \equiv 1 \pmod p$. Because the least power of $b$ that is equivalent to $1$ modulo $p$ is $p-1$, $p -1 \mid j \cdot \frac{p-1}{2}$. Thus $j$ must be even, and $\frac{j}{2}$ is an integer. Thus we have that $(b^{\frac{j}{2}})^2 \equiv a \pmod p$, and thus $a$ must be a quadratic residue.
	
\end{proof}

\par You may have noticed that the Legendre Symbol is only defined for a prime $p$. We define an analogue for composites, \textit{Jacobi's Symbol}, $\left( \frac{a}{n} \right)$, as $\prod_i \left( \frac{a}{p_i} \right) ^{\alpha_i}$ for $n$ an odd integer, $(a,n) = 1$, and $n = \prod_i p_i^{\alpha_i}$. 

\par Recalling that we never addressed the case of the product of two quadratic non residues of a composite number, it makes intuitive sense that the Jacobi symbol occasionally incorrectly takes the value $1$ for a quadratic non-residue, given it is the product of two non-residues. However, it never incorrectly identifies a quadratic residue.  Thus it can be used to quickly determine that something is not a quadratic residue modulo $n$, but cannot prove something as a quadratic residue mod $n$.

%\subsection*{Modules}

%\begin{definition}
%A set of numbers $M$ is called a module if for every $x$ and $y$ in $M$ $x+y$ and $x-y$ are also in $M$.
%\end{definition}
%We can view modules as sets closed under subtraction, because if $x-y \in M$, $x - x = 0 \in M$, and consequently $0 - y = - y \in M$ so $x - (-y) = x +y \in M$. Examples of modules include the set of integers, and the set of all integer multiples of a real number $\alpha$.

%\par We have the following theorem for a module $M \subseteq \mathbb Z$.
%\begin{theorem}
%All elements of a module $M \subseteq \mathbb Z$, $M$ not equal to $\{0\}$, are multiples of an integer $d$, the smallest positive %integer contained in $M$.
%\end{theorem}
%\begin{proof}
%Let $d$ be the smallest positive element of $M$. All multiples of $d$ must be in $M$ as well, because we can continually sum $d$ to %obtain positive multiples, and $-d$ to obtain negative multiples. Also, $d - d = 0 \in M$. Suppose that there is some integer $x$ in %$M$ that is not a multiple of $d$. Thus $ nd < x < (n+1)d$ for some $n \in \mathbb Z$. But then $x - nd \in M$ as well, and that %element is less than $d$, which condradicts the minimality of $d$. So there is no integer in $M$ that is not a multiple of the least %positive integer $d$.
%\end{proof}

%\par It should be noted that the smallest module $M$ containing two integers $a$ and $b$ is the one generated by $d = (a,b)$. The proof is trivial.




\section*{Primality Testing}\label{Primality}

% Introduction to the subject
\par First, we will treat the various methods of determining whether or not a number is prime, or, as is usually easier, determining
whether a number is composite. At first glance, it seems strange to have methods for such a thing that are distinct from factorization methods - after all, if you can obtain the prime factorization of a number, it is trivial to determine primality from that factorization. However, factorization methods are in general very computationally expensive compared to some of the methods we will examine in this section. 


%%%%%%%%%%%%%%%%%%%%%%%%%%%%%%%%%%%
\subsection*{ Primality Tests and Composite Tests }

\par Any viable computational method for testing primality explicitly is composed of a condition on a number $n$ that, when met, necessitates that $n$ is prime. Thus if the condition is not met, $n$ is composite. Such tests are certainly extremely convenient to directly determine the primality of $n$, but unfortunately methods of this form are usually a combination of very complex and restricted to $n$ of a particular form or within some bounded range.

\par In addition to these primality tests, which never fail on determining primality, we also have many tests that are comparatively simple and computationally efficent, which occasionally fail to identify primes, but never indicate that a composite number is prime. We will call these tests, which never fail on determining compositeness, compositeness tests.

\par It is of the utmost importance to note, before moving on, that while the conditions of a primality or compositeness test being met guarantees that the number is either prime or composite respectively, and a failed primality test proves compositeness, a failed compositeness test does not necessitate that the number is prime.

%%%%%%%%%%%%%%%%%%%%%%%%%%%%%%%%%%%
\subsection*{ The Sieve of Eratosthenes }

\par Sieving is a process where a series of operations are applied to every number in a large, regularly spaced set of integers in order to find numbers with certain characteristics.

\par The Sieve of Eratosthenes is among the first algorithmic methods for factorization and primality testing. Admittedly, it is very crude, but the theoretical basis of the method gives us some of the fundamental ideas for developing further primality tests. Additionally, the Sieve of Eratosthenes also can be used for algorithmic factorization of a number - as such, we will treat it rather lightly here and revisit it in more detail when we examine factorization methods.

\par The core use of the Eratosthenes' sieve in primality testing stems from the idea of \textit{trial division}, where given an integer $n$, we attempt to divide $n$ by every integer that could possibly be a factor. If the number divides $n$, then it is a factor of $n$, and because possible factors must be bounded above by $n$ itself, we are guaranteed a finite number of computations. In fact, we need only test factors less than $\sqrt{n}$, recovering greater factors as the quotient of a successful division by a lesser factor. Additionally, after each successful division, only the quotient need be tested further, resulting in easier computation as more factors are found. None of these facts are particularly reassuring computationally, but they provide some background for the use of our sieve.

\par The sieve can be thought of as a list or array of $N - 1$ consecutive integers, beginning with $2$. We recognize that $2$ is prime, and thus every multiple of $2$ is composite. We then remove $4$,$6$, etc. from our list, and are left with $2$ and the odd integers. We then recognize $3$ as prime, and remove all multiples of $3$ in the same way. $5$ has not been removed from the list, and so it is prime. We then repeat our method. Continuing this way until the end of the set has been reached, we have constructed a set of all prime numbers up to $N$.

\par Note that the sieve has other modifications that can be made to find things like -- the least prime factor of each composite up to $N$, or even the complete factorizations of the numbers in the sieve.

\par It is a fair observation that this is not at all an individualized test, but rather construction of a set that will prove a number $n$ prime if $n$ is an element of the set, and composite if $n < N$ and $n$ is not in the set. This is not ideal, but combined with our method of trial division, and given that $76\%$ of odd integers have a prime factor less than $100$, precompiling a list of primes up to a certain bound and performing trial divisions of only those prime numbers on a given $n$ can prove many numbers composite without having to resort to more rigorous primality tests.

%%%%%%%%%%%%%%%%%%%%%%%%%%%%%%%%%%%
\subsection* {Fermat's Theorem and Resulting Methods}

\par Among the many limitations of Eratosthenes' sieve is its lack of precision - it is impossible to simply prove a given number $n$ prime or composite with the sieve, you must construct the entire set of primes up to $n$ in the worst case to say anything about $n$ at all. We will next move to our very first formal test, a compositeness test based on a theorem of Fermat.

\par In passing through the list, we are adding $p$ to find the next multiple. In addition to that, we noted earlier that we are guaranteed a finite number of computations. Therefore, this sieve has the benefit of not being particularly computationally intensive. In practice, the problem is the amount of memory it can consume. This is dependent upon the size of the list of numbers. There are work-arounds for this, like array segmentation, but this may affect the efficiency of the sieve.

%%%%%%%%%%%%%%%%%%%%%%%%%%%%%%%%%%%%	
\subsubsection*{Fermat's Theorem}

\newtheorem*{fermattheorem}{Fermat's Theorem}
\begin{fermattheorem}
If a positive integer $p$ is a prime and $(a,p) = 1$, then
	$$a^{p-1} \equiv 1 \pmod p$$
\end{fermattheorem}

\par Fermat's theorem cannot be used as a primality test, as we will examine further when we define pseudoprimes, but we can use the converse of the theorem to develop a test for compositeness.

\newtheorem*{fermatconverse}{Converse of Fermat's Theorem}
\begin{fermatconverse}
If $n$ and $a$ are positive integers such that $(a,n) = 1$ and
	$$a^{n-1} \not\equiv 1 \pmod n,$$
	then $n$ is not a prime. Hence $n$ is composite.
\end{fermatconverse}

%%%%%%%%%%%%%%%%%%%%%%%%%%%%%%%%%%%%
\subsubsection*{ Pseudoprimes and Carmichael Numbers }
There are composite numbers that will behave similarly to primes when Fermat's theorem is applied to them for certain $a$. We will call these composites \textit{Fermat pseudoprimes}.
\begin{definition}
A \textit{ Fermat pseudoprime} base $a$ is an odd composite number $m$ for which 
$$ a^{m-1} \equiv 1 \pmod m$$  
holds. 
\end{definition}

 \par In general, a \textit{pseudoprime} is a composite number that behaves like a prime in the context of a primality or compositeness test. We will see other pseudoprimes in the future. If it were somehow possible to characterize all Fermat pseudoprimes for a base $a$, we could then use Fermat's Theorem with base $a$ as a primality test - this is approachable for fixed length numbers.
 
 \par The question arises - can all Fermat pseudoprimes be detected through a change of base? The answer, unfortunately, is no.   There exist numbers that will be Fermat pseudoprimes base $k$ for all valid $k$, and these numbers motivate the following definition.
 
 \begin{definition}
  A composite number $n$ such that $a^{n -1} \equiv 1 \pmod n$ for all $a$ relatively prime to $n$ is called a \textit{Carmichael Number}.
 \end{definition}
 
 A Carmichael number, the smallest of which is $561$, will never be revealed as composite by a Fermat's Theorem test. The existence of Carmichael numbers are a serious hindrance of the Fermat's theorem method, although fortunately they can be characterized.

\begin{theorem} $n$ is a Carmichael number if and only if $p-1$ divides $n-1$ for every prime factor of $n$, $n$ is composite, and $n$ is squarefree.
\end{theorem}

\begin{proof}
First, suppose that $n$ is a composite number that square free, and $p-1 \mid N-1$ for every prime factor $p$ of $n$, and that $a$ is an arbitrary integer such that $(a,n) = 1$.
\par $n$ is square free, so $n = p_1 p_2 p_3 \ldots p_k$. Because $a$ is relatively prime to $n$, $a$ is also relatively prime to $p_i$ for all $i \in {1, \ldots, k}$. Thus, $a^{p-1} \equiv 1 \pmod{p_i}$ for all $i$ by Fermat's Theorem. However, because $p -1 \mid n -1$, $a^{n-1} = (a^{p-1})^t \equiv (1)^t = 1 \pmod{p_i}$ for some $t \in \mathbb Z$. Because $p_i \mid a^{n-1} - 1$ for all $i$, and all $p_i$ are distinct primes, $n = p_1p_2p_3 \ldots p_k \mid a^{n-1} - 1$. Thus $a^{n-1} \equiv 1 \pmod N$ for any arbitrary $a$ relatively prime to $n$, and $n$ is a Carmichael number.
\par
We now suppose that a composite $n$ is a Carmichael number, so $a^{n-1} \equiv 1 \pmod n$ for all $a$ relatively prime to $n$. $n = p_1^{\alpha_1}\cdot \ldots p_k^{\alpha_k}$, and so $p_i^{\alpha_i} \mid a^{n-1} - 1 $ and $(a,p_i) = 1$ for all $i \in \{1, \ldots, k\}$. Let $r$ be an integer such that $N = p_i^{\alpha_i} r$. Obviously, $p_i^{\alpha_i}$ and $r$ are relatively prime, so by the chinese remainder theorem we have a unique solution of the system
	$$x \equiv a_1 \pmod{p_i^{\alpha_i}}$$
	$$x \equiv 1 \pmod{r}$$
where $a_1$ is a primitive root of $p_i^{\alpha_i}$. 
\par Suppose that $\alpha_i = 1$. $x$ is coprime to both $p_i$ and $r$, so we have $x \equiv a \pmod{p_i \cdot r = n}$, and so $x^{n-1} \equiv 1 \pmod n$, and so $x^{n-1} \equiv 1 \pmod p_i$ as well. However, $x^{p_i - 1} \equiv 1 \pmod{p_i}$, and $p_i - 1$ is the least power of $x$ such that $x$ is equivalent to $1$ modulo $p_i$, and so we have $p_i -1 \mid N - 1$ for all $i$. 
\par Now suppose that $\alpha_i >1$ for some $i$, or that $n$ is not squarefree. $x^{n-1} \equiv 1 \pmod{p_i^{\alpha_i}}$, and by Euler's theorem, $x^{\phi(p_i^{\alpha_i})} = x^{p_i^{\alpha_i - 1}(p_i - 1))}  \equiv 1 \pmod{p_i^{\alpha_i}}$. Thus because $p_i^{\alpha_i -1}(p_i - 1)$ is the least power of $a_1$ equivalent to $1$ modulo $pi^{\alpha_i}$, $p_i^{\alpha_i -1}(p_i - 1) \mid n - 1$. However, $p_i^{\alpha_i - 1} \mid n - 1$ is a contradiction, because $(n, n-1) = 1$ and $p_i^{\alpha_i - 1} \mid n$. Thus $n$ must also be squarefree, and we are done. 
\end{proof}


\subsubsection*{Improvements due to Eulers Criterion}

There is clear motivation to improve our method, if possible, to avoid the problems presented by Carmichael numbers. One tool that we have at our disposal is the \textit{Euler Criterion for Quadratic Residues}, which is is stated below and proven in the Background %%ref 
section.

\newtheorem*{eulerconverse}{Euler's Criterion as a Compositeness Test}

\begin{eulercriterion}
If $p$ is an odd prime and $(a,p) = 1$, 
$$a^{\frac{n-1}{2}} \equiv \pm 1 \pmod p$$
\end{eulercriterion}


This criterion's logical converse, much like Fermat's theorem, gives us a test for compositeness.
\begin{eulerconverse}
If $n$ is odd, $(a,n) = 1$, and
	$$a^{\frac{n-1}{2}} \not\equiv \pm 1 \pmod n$$
	then $n$ is a composite number.
\end{eulerconverse}

To strengthen this compositeness test, we use \textit{Jacobi's symbol}, $( \frac{a}{n})$, over Legendre's symbol.

\begin{eulerconverse}
If $n$ is odd, $(a,n) = 1$, and
	$$a^{\frac{n-1}{2}} \not\equiv \pm 1 \pmod n$$
	then $n$ is a composite number.
	
\par If $a^{\frac{n-1}{2}} \equiv \pm 1 \pmod n$ and
	$$a^{\frac{n-1}{2}} \not\equiv (\frac{a}{n}) \pmod n$$
	then $n$ is composite.
	
\par However, if $a^{\frac{n-1}{2}} \equiv (\frac{a}{n}) \pmod n$ holds, the test is inconclusive.
\end{eulerconverse}

As the last line of our definition implies, this test also has pseudoprimes associated with it. We will call these \textit{Euler pseudoprimes}.
\subsubsection*{Euler Pseudoprimes and Strong Pseudoprimes}
We define an Euler pseudoprime in an analogous way to our definition of a Fermat pseudoprime.
\begin{definition}
Let $n$ be an odd composite number such that
	 $$a^{\frac{n-1}{2}} \not\equiv (\frac{a}{n}) \pmod n$$ 
for some $a$ relatively prime to $n$.  Then $n$ is an \textit{Euler pseudoprime} base $a$.
\end{definition}

Unfortunately, there are some numbers that are both Carmichael numbers and Euler pseudoprimes. An example of one such number is $1729$ - %% flesh out this example later
\par Despite this disappointing fact,  many Carmichael numbers are revealed as composite by Euler's criterion, and additionally, there is no analogue for Carmichael numbers in Euler pseudoprimes. If enough bases are tested, we will eventually prove an Euler pseudoprime composite.

\par To close this section, we present a final type of pseudoprime.
\begin{definition}
An odd composite number $n$ with $n-1 = d\cdot2^s$, $d$ odd, is called a \textit{strong pseudoprime} for base $a$ if either
$a^d \equiv 1 \pmod n$ or $a^{d \cdot 2r} \equiv -1 \pmod n$, for some $r \in \{0, 1, \ldots, s-1\}$. 
\end{definition}

\par This test for pseudoprimes is intended, much like Euler pseudoprimes, to eliminate the problem of Carmichael numbers in the Fermat
Compositeness test. Indeed, any Fermat pseudoprime will be eventually proven composite by the strong pseudoprime test if enough 
bases $a$ are tested.

\par The motivation for the definition is as follows - Any Fermat pseudoprime $n$ to base $a$ will satisfy the equivalence
	$$a^{n-1} - 1 \equiv 0 \pmod n$$
Because we assume that $n$ is odd if it is being tested for primality, $n = 2m+1$ for some integer $m$ and we have
	$$a^{2m} - 1 = (a^m - 1)(a^m+1) \equiv 0 \pmod(n)$$
If $n$ is a prime, it must divide one of these factors, but it cannot divide both because then it would need to divide all linear
combinations of them, including $(a^m - 1) - (a^m + 1) = -2$. Therefore we have 
	$$a^m \equiv \pm 1 \pmod n$$
We can write $n$ as $2^\alpha k + 1$, where $k$ is odd, and have 
	$$a^{n-1} - 1 = (a^k - 1)(a^k + 1)(a^{2k} + 1) \ldots (a^{2^{\alpha - 1} k} +1)$$
	
\par If $n$ divides exactly one of these factors but is composite, then it is a strong pseudoprime. Interestingly, if a number is
a strong pseudoprime to the base $a$, it will also be a Euler pseudoprime to $A$.





%%%%%%%%%%%%%%%%%%%%%%%%%%%%%%%%%%%%%%%%%%%%%%%%%%%%%%%
%%%%%%%%%%%%%%%%%%%%%%%%%%%%%%%%%%%%%%%%%%%%%%%%%%%%%%%
%%%%%%%%%%%%%%%%%%%%%%%%%%%%%%%%%%%%%%%%%%%%%%%%%%%%%%%
%%%%%%%%%%%%%%%%%%%%%%%%%%%%%%%%%%%%%%%%%%%%%%%%%%%%%%%



\subsection*{Proving Primality}

\par In the previous section, all the methods we examined could only possibly prove compositeness - because we are unable to test
infinitely many bases in an Euler or Strong Pseudoprime test, there is no way for us to computationally prove that a number is prime using
these methods, only to say that it is very likely to be prime, or prove that it is composite. 
\par In this section, we examine some methods of proving primality. These primality tests are much more complicated computationally and theoretically than compositeness tests, and unfortunately usually depend on factorization of a number, which is a slow and laborious 
process. However, they are theoretically motivating, despite not being as generally popular as more modern primality tests.

\subsubsection*{Lehmer's Theorem}

\par The first method of actually proving primality that we will look at is based on a theorem by Lucas, and proven by Lehmer.

\newtheorem*{lehmer}{Lehmer's Theorem}
\begin{lehmer}
Suppose $n-1 = \prod^n_{j=1} q_j^{\beta_j}$, with all $q_j$ distinct primes. If an integer $a$ exists such that 
	$$a^\frac{n-1}{q_j} \neq 1 \pmod n \text{ for all } j = 1, \ldots, n$$
	$$\text { and such that }$$
	$$a^{n-1} \equiv 1\pmod n \text{, }$$
then $n$ is a prime number.
\end{lehmer}
We note that this theorem is an extension of Fermat's theorem. 
\begin{proof}
Consider $a^k \equiv 1 \pmod n$. The smallest such $k$ divides all possible $k$, including $n-1$. However, every divisor $k$ not equal to $n-1$ of $n-1$ divides at least one $\frac{n-1}{q_j}$. Thus if $k$ is  less than $n-1$,  $a^{\frac{n-1}{q_j}} \equiv a^{tk}\equiv 1 \pmod n$ will hold for some $q_j$. This contradicts the assumptions of the theorem, and so $k = n-1$. However, by Euler's theorem, we know that $a^{\phi(n)} \equiv 1 \pmod n$ for all $a$ relatively prime to $n$, and that $\phi{n} < n-1$ for all composite numbers. If $\phi{n} < k$, it contradicts the assumption that $k$ is the smallest power of $a$ equivalent to $1$, and thus $\phi{n} = k = n-1$, and $n$ must be prime.
\end{proof}

\par From our knowledge of primitive roots, we know that any primitive root will be a suitable $a$ for Lehmer's Theorem. The problem inherent in this is that there is not an efficient, deterministic method to find primitive roots, or even quadratic non-residues. If $n$ is prime, then we will have $\phi(\phi(n)) = \phi(n-1) = (n-1) \prod (1- \frac{1}{q_j}$ primitive roots, but as the equation shows, if $n-1$ has many factors, especially many small factors, primitive roots become a smaller proportion of possible $a$ values.

 \par Fortunately, Selfridge provides us with a relaxed version of Lehmer's theorem such that we do not need a primitive root to prove $n$ prime.
\begin{theorem}
Suppose $n-1 = \prod^n_{j=1} q_j^{\beta_j}$, with $q_j$ all distinct primes.  If for every $q_j$, there is an $a_j$ such that
	$$a_j^{\frac{n-1}{q_j} \not\equiv 1 \pmod n}$$ 
	$$\text{ while }$$
	$$a_j^{n-1} \equiv 1 \pmod n \text{, }$$
then $n$ is a prime.
\end{theorem}
%% possibly reference a text where someone can find the proof
\par This allows us to perform tests with multiple bases, and prove primality usually much more quickly than Lehmer's theorem applied directly, especially for primes where the least primitive root is large.
\par However, there is still a major stumbling block in practical applications of Lehmer's Theorem - the need to factorize $n -1$. Luckily, it is possible to relax the conditions of Lehmer's theorem in a way that allows us to only partially factorize $n-1$.

\begin{theorem}
Let $n - 1 = r \cdot f$, where $f$ is the factorized part of $n-1$ and $r$ is the unfactorized or remaining part, $(r,f) = 1$,$r < f$. $F = \prod^n_{j=1} q_j^{\beta_j}$, with all $q_j$ distinct primes.
\par If an integer $a$ exists such that
	$$\text{gcd}(a^{\frac{n-1}{q_j}} - 1, N) = 1 \text{ for all } j \text{, }$$
	$$\text{ and }$$
	$$a^{n-1} \equiv 1 \pmod n \text{, }$$
then $n$ is a prime number.
\end{theorem}

% There is an analogous theorem for factorization, the Lehmer-Pocklington Theorem.
%%possibly reference the lehmer-pocklington theorem here, if jodie is covering it

\subsubsection*{P\'{e}pin's and Proth's Theorems}
Another possible sidestep of the need to factor $n-1$ arises when $n$ happens to be a Fermat number, or a number of the form $2^{2^r} + 1$ (These happen to be the only sums of a power of 2 and 1 that can be prime). In this case, $n - 1$ has the minimum possible number of distinct prime factors, and thus is very suited to a Lehmer's Theorem test. In fact, for a Fermat number $n$, the requirements of Lehmer's Theorem become
	$$a^{\frac{n - 1}{2}} = a^{2^{2^{r-1}}} \not\equiv 1 \pmod n $$
	$$\text{ and } $$
	$$a^{n - 1} = a^{2^{2^{r}}} \equiv 1 \pmod n \text{, }$$
which essentially reduces to 
	$$x^2 \equiv 1 \pmod n$$
	$$\text{ and }$$
	$$x \not \equiv 1 \pmod n$$
where $x = a^{2^{2^{r-1}}}$.

If $n$ is prime, then $x^2 \equiv 1 \pmod n$ will have exactly two solutions - $x \equiv \pm 1 \pmod n$. $x \equiv 1 \pmod n$ violates the conditions of Lehmer's theorem, and so $x \equiv -1 \pmod n$ when $n$ is prime. This motivates a need for an $a$ such that 
	$$x \equiv a^{\frac{n - 1}{2}} \equiv -1 \pmod n$$
	
\par Noting that Euler's Criterion gives us that $a$ must be a quadratic non-residue of $n$. The fact that $3$ is a quadratic non-residue of all primes of the form $12n \pm 5$ and the observation that $2^{2^k} \equiv 4 \pmod 12$ for all $k$ motivates the idea of (but does not prove) P\'{e}pin's Theorem, which is stated below.

\newtheorem*{pepin}{P\'{e}pin's Theorem}
\begin{pepin}
A Fermat number $n = 2^{2^r} + 1$, $n \geq 1$ is prime if and only if 
	$$ a^{\frac{n - 1}{2}} \equiv  - 1 \pmod n$$
\end{pepin}

It is worth mentioning that the binary computer makes the division by $n$ necessary to reduce powers of $a$ modulo $n$ very computationally simple, due to $n$ being a power of $2$ plus $1$ and therefore very simply expressed in binary. This makes P\'{e}pin's test a very attractive test for Fermat number primes. 

With both  P\'{e}pin's Theorem and relaxed version of Lehmer's theorem, we obtain Proth's Theorem, stated below.

\newtheorem*{proth}{Proth's Theorem}
\begin{proth}
Suppose $n$ is of the form $n = m \cdot 2^k + 1$, with $2^k > m$ and $m$ odd. If there exists an integer $a$ such that 
	$$a^{\frac{n-1}{2}} \equiv -1 \pmod n \text{, }$$
then $n$ is prime.
\end{proth}

\par There is another way to possibly ease the difficulty of factorizing $n - 1$, namely by using a different compositeness test than Fermat's theorem, that allows us to attempt to factorize a number other than $N - 1$, which may prove easier to factor. For example, primality tests based on Lucas Sequences give an analogue to Lehmer's theorem in quadratic fields, where $N+1$ may be factored instead of $N-1$. The theory of quadratic fields is beyond the scope of this paper, and so we do not address Lucasian or similar methods, but their existence is worth noting.


%%%%%%%%%%%%%%%%%%%%%%%%%%%%%%%%%%%%%%%%%%%%%%%%%%%%%%%
%%%%%%%%%%%%%%%%%%%%%%%%%%%%%%%%%%%%%%%%%%%%%%%%%%%%%%%
%%%%%%%%%%%%%%%%%%%%%%%%%%%%%%%%%%%%%%%%%%%%%%%%%%%%%%%
%%%%%%%%%%%%%%%%%%%%%%%%%%%%%%%%%%%%%%%%%%%%%%%%%%%%%%%

\subsection*{Modern Primality Tests}

%% We will treat these very lightly 
\par Of the primality tests covered in the previous section, none are of practical use for all numbers. Computationally viable methods such as P\'{e}pin's Theorem only apply to primes of a certain form, and the need for factorization and primitive roots cripples Lehmer's Theorem computationally. Fortunately, there exist more suitable methods used today in primality testing. 
\par These tests are, in general, quite theoretically and algorithmically complicated, and so we give only a brief introduction to the ideas behind modern primality tests and do not attempt to rigorously present them.

\subsubsection*{The Jacobi Sum Primality Test}

\par The foundation of the Jacobi Sum Primality test is in its use of cyclotomic number fields. Put in the simplest terms possible, the test uses information from a combination of generalized pseudoprime tests based on Fermat's theorem, like the ones addressed earlier in this paper, in cyclotomic rings. The results of these tests is then used to construct a sieve for the possible prime divisors of a number $n$ until in the end it is proven to be its own sole prime divisor, or a prime number. 

 \par Also called the Adleman-Pomerance-Rumely primality test,  the Jacobi Sum Primality test does not use random numbers like many of the more efficient primality tests, allowing it to be a deterministic primality test in exchange for computational superiority.

\subsubsection*{Elliptic Curve Primality Testing}

Elliptic curve primality testing is based on the use of the properties of the group of points modulo $n$ on an elliptic curve. Unlike the Jacobi Sum Test, Elliptic Curve testing is probablilistic, and therefore may theoretically fail. That isn't to say that it gives the wrong answer - that would make it not a primality test - but it is possible for the algorithm to run indefinitely, never proving even a prime $n$ prime.  Despite this, elliptic curve methods are among the fastest and most popular primality testing algorithms used today.
\par Stated very generally, elliptic curve methods use elliptic curves generated either by random integers in the case of the Goldwasser-Killian algorithm or by methods guaranteed to generate curves that will be computationally simpler for the Atkin-Morain test. The number of integer points on the curve modulo $n$ if $n$ is prime is then computed, and then tested using a probable prime $q$ determined by conditions based on $n$ and the elliptic curve selected, which combined with the criteria of the test will either prove $n$ composite, or allow us to prove $n$ prime given that we can verify the primality of $q$. Of course, the method can then be recursively applied to $q$, each time resulting in a smaller probable prime (and of course applying less computationally taxing compositeness tests to each new probable prime before returning to elliptic curves). With this recursion, we will either end up proving the compositeness of one and therefore all of our probable primes, or arriving at a well known prime and proving them all prime. 
\par The test depends on the fact that the only non-invertible equivalence class modulo a prime is $0$, or the class containing multiples of the prime itself. The computations necessary for proving a number $n$ prime using an elliptic curve cannot be performed on a non-invertible element modulo $n$,  but if we end up encountering such an element outside of the multiples of $n$, we have shown that $n$ cannot be prime and the test can conclude. 


\section*{Factorization}\label{Factorization}

\section*{Conclusion}\label{Conclusion}

\newpage

\nocite{*}
\bibliographystyle{plainnat}         
\bibliography{references}




%%
%% The following sections are something we should fill in together at some point,
%% They are not important right now but are nice to have, I'm not really sure what is relevant to put there.
%%	-Margaret
%%

\section*{About the authors:}


\par This paper was written to fulfill the requirements of Math 371, Number Theory, offered by Dr. Anurag Agarwal
in the Spring 2016 semester at Rochester Institute of Technology.
\subsection*{Margaret Dorsey}
   Computational Mathematics
   \\Game Design and Development
   \\Computer Science
   \\
   \\ med7068@rit.edu
  \\
  \\ Responsible for all proofs provided directly in the paper in both the Background and Primality sections.
  \\
  \\ Author of the Introduction, Background, Primality, and  co-author of the History section.

\subsection*{Jodie Miu}
   Computer Science
   \\
   \\ jm7481@rit.edu 
   \\
   \\ Author of the Factorization section, and co-author of the History section.
   
   
%%%%%%%%%%%%%%%%%%%%%%%%%%%%%%%%%%
%% 
%%	Some notes and explanations
%%
%%%%%%%%%%%%%%%%%%%%%%%%%%%%%%%%%%	

%%%%%%%%%%%%%%
%%
%% Matrix Notation - don't know if we'll end up using it, but it is kind of strange so here's an example
%%
%%%%%%%%%%%%%%


%% $$
%%  \left[
%% 	\begin{array}{rrrrrrrr}
 %% 		0 & -1 & 0 & 1 & 0 & 0 & 0 & 0  \\
 %% 		1 & 0 & 0 & 0 & -1 & 0 & 0 & 0  \\
 %% 		0 & 0 & \alpha & \alpha & -\alpha & -\alpha & 0 & 0  \\
 %% 		\alpha & \alpha & 0 & 0 & 0 & 0 & -\alpha & -\alpha  \\
 %% 		0 & 0 & 1 & 0 & 0 & 0 & -1 & 0  \\
 %% 		0 & 0 & 0 & 0 & 0 & 1 & 0 & -1
 %% 	\end{array}
 %% \right]
 %%	$$
 
%%%%%%%%%%%%%%
%%
%% Figures - this is how you use a figure made from Mathematica or Maple in .eps format. The .eps file must be in the same directory
%% .eps file must be pure ascii mode, no preview, no thumbnail
%% I have both maple and mathematica if we need figures, so don't bother to go and buy a license.
%%		-Margaret
%%
%%%%%%%%%%%%%%

%%	\begin{figure}[htb]
%%     \centering
%%     \includegraphics[width=2.5in]{para01.eps}
%%     \caption{$y = x^{2}$\label{ParabFig}}
 %%	 \end{figure}

\end{document}
