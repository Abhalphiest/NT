%%%%%%%%%%%%%%%%%%%%%%%%%%%%%%%%%%%%%%%%%%%%%%%%%%%%%%%
% % 
% % 	Introduction Draft
%%
% %	 Math 371: 		Number Theory, Spring 2016
% % 	Author(s): 		Margaret Dorsey
% % 	Last Updated: 	04 / 16 / 2016	(see gitlog.txt for revision history)
% % 
%%%%%%%%%%%%%%%%%%%%%%%%%%%%%%%%%%%%%%%%%%%%%%%%%%%%%%%
% % Title and author(s)
%%%%%%%%%%%%%%%%%%%%%%%%%%%%%%%%%%%%%%%%%%%%%%%%%%%%%%%


\documentclass{article}
\usepackage{graphicx}
\usepackage{amsthm} %theorem formatting
\usepackage{amsfonts} % fonts
\newenvironment{intro}
  {\renewcommand\abstractname{Introduction}\begin{abstract}}
  {\end{abstract}}



%%%%%%%%%%%%%%%%%%%%%%%%%%%%%%%%%%%%%%%%%%%%%%%%%%%%%%%
% % There are a lot of commands for theorems, declarations, etc.
% %
% % Here are some common ones. 
%%		-Margaret
%%%%%%%%%%%%%%%%%%%%%%%%%%%%%%%%%%%%%%%%%%%%%%%%%%%%%%%
%%
\newtheorem*{theorem}{Theorem}
%%\newtheorem{lemma}{Lemma}
%%\newtheorem{proposition}{Proposition}
%%\newtheorem{scolium}{Scolium}   %% Maybe common was an overstatement..
\newtheorem*{definition}{Definition}
%%\newenvironment{proof}{{\sc Proof:}}{~\hfill QED}
%%\newenvironment{AMS}{}{}
%%\newenvironment{keywords}{}{}
%%
%%%%%%%%%%%%%%%%%%%%%%%%%%%%%%%%%%%%%%%%%%%%%%%%%%%%%%%

\begin{document}
\newpage


%%%%%%%%%%%%%%%%%%%%%%%%
%%
%%  \section{ sectionName }
%%	\subsection{ name}
%%	\subsubsection{name}
%%	etc..
%%  Use sectioning commands for headings. Often longer articles are divided into a few sections.
%%
%%%%%%%%%%%%%%%%%%%%%%%%%%%%%%%%
%%
%%  use /par to start a new paragraph, // for a simple line break.
%%
%%  You must use labelling commands, e.g. \label, \ref,
%%  \bibitem, and \cite to refer to sections of your document, such as
%%  see Section~\ref{sectionname}, see Figure~\ref{figurename}, or bibliography entries,
%%  such as see~\cite{citationname}.  Otherwise the look of the numbers, and sometimes the numbers
%%  themselves, will get messed up.
%%
%%%%%%%%%%%%%%%%%%%%%%%%%%%%%%%%%

\section{ Primality Testing }

% Introduction to the subject
\par First, we will treat the various methods of determining whether or not a number is prime, or, as is usually easier, determining
whether a number is composite. At first glance, it seems strange to have methods for such a thing that are distinct from factorization methods - after all, if you can obtain the prime factorization of a number, it is trivial to determine primality from that factorization. However, factorization methods are in general very computationally expensive compared to some of the methods we will examine in this section. 


%%%%%%%%%%%%%%%%%%%%%%%%%%%%%%%%%%%
\subsection{ Primality Tests and Composite Tests }

\par Any viable computational method for testing primality explicitly is composed of a condition on a number $n$ that, when met, necessitates that $n$ is prime. Thus if the condition is not met, $n$ is composite. Such tests are certainly extremely convenient to directly determine the primality of $n$, but unfortunately methods of this form are usually a combination of very complex and restricted to $n$ of a particular form or within some bounded range.

\par In addition to these primality tests, which never fail on determining primality, we also have many tests that are comparatively simple and computationally efficent, which occasionally fail to identify primes, but never indicate that a composite number is prime. We will call these tests, which never fail on determining compositeness, compositeness tests.

\par It is of the utmost importance to note, before moving on, that while the conditions of a primality or compositeness test being met guarantees that the number is either prime or composite respectively, and a failed primality test proves compositeness, a failed compositeness test does not necessitate that the number is prime.

%%%%%%%%%%%%%%%%%%%%%%%%%%%%%%%%%%%
\subsection{ The Sieve of Eratosthenes }

\par Sieving is a process where a series of operations are applied to every number in a large, regularly spaced set of integers in order to find numbers with certain characteristics.

\par The Sieve of Eratosthenes is among the first algorithmic methods for factorization and primality testing. Admittedly, it is very crude, but the theoretical basis of the method gives us some of the fundamental ideas for developing further primality tests. Additionally, the Sieve of Eratosthenes also can be used for algorithmic factorization of a number - as such, we will treat it rather lightly here and revisit it in more detail when we examine factorization methods.

\par The core use of the Eratosthenes' sieve in primality testing stems from the idea of \textit{trial division}, where given an integer $n$, we attempt to divide $n$ by every integer that could possibly be a factor. If the number divides $n$, then it is a factor of $n$, and because possible factors must be bounded above by $n$ itself, we are guaranteed a finite number of computations. In fact, we need only test factors less than $\sqrt{n}$, recovering greater factors as the quotient of a successful division by a lesser factor. Additionally, after each successful division, only the quotient need be tested further, resulting in easier computation as more factors are found. None of these facts are particularly reassuring computationally, but they provide some background for the use of our sieve.

\par The sieve can be thought of as a list or array of $N - 1$ consecutive integers, beginning with $2$. We recognize that $2$ is prime, and thus every multiple of $2$ is composite. We then remove $4$,$6$, etc. from our list, and are left with $2$ and the odd integers. We then recognize $3$ as prime, and remove all multiples of $3$ in the same way. $5$ has not been removed from the list, and so it is prime. We then repeat our method. Continuing this way until the end of the set has been reached, we have constructed a set of all prime numbers up to $N$.

\par Note that the sieve has other modifications that can be made to find things like -- the least prime factor of each composite up to $N$, or even the complete factorizations of the numbers in the sieve.

\par It is a fair observation that this is not at all an individualized test, but rather construction of a set that will prove a number $n$ prime if $n$ is an element of the set, and composite if $n < N$ and $n$ is not in the set. This is not ideal, but combined with our method of trial division, and given that $76\%$ of odd integers have a prime factor less than $100$, precompiling a list of primes up to a certain bound and performing trial divisions of only those prime numbers on a given $n$ can prove many numbers composite without having to resort to more rigorous primality tests.

%%%%%%%%%%%%%%%%%%%%%%%%%%%%%%%%%%%
\subsection {Fermat's Theorem and Resulting Methods}

\par Among the many limitations of Eratosthenes' sieve is its lack of precision - it is impossible to simply prove a given number $n$ prime or composite with the sieve, you must construct the entire set of primes up to $n$ in the worst case to say anything about $n$ at all. We will next move to our very first formal test, a compositeness test based on a theorem of Fermat.

\par In passing through the list, we are adding $p$ to find the next multiple. In addition to that, we noted earlier that we are guaranteed a finite number of computations. Therefore, this sieve has the benefit of not being particularly computationally intensive. In practice, the problem is the amount of memory it can consume. This is dependent upon the size of the list of numbers. There are work-arounds for this, like array segmentation, but this may affect the efficiency of the sieve.

%%%%%%%%%%%%%%%%%%%%%%%%%%%%%%%%%%%%	
\subsubsection{Fermat's Theorem}

\newtheorem*{fermattheorem}{Fermat's Theorem}
\begin{fermattheorem}
If a positive integer $p$ is a prime and $(a,p) = 1$, then
	$$a^{p-1} \equiv 1 \pmod p$$
\end{fermattheorem}

\par Fermat's theorem cannot be used as a primality test, as we will examine further when we define pseudoprimes, but we can use the converse of the theorem to develop a test for compositeness.

\newtheorem*{fermatconverse}{Converse of Fermat's Theorem}
\begin{fermatconverse}
If $n$ and $a$ are positive integers such that $(a,n) = 1$ and
	$$a^{n-1} \not\equiv 1 \pmod n,$$
	then $n$ is not a prime. Hence $n$ is composite.
\end{fermatconverse}

%%%%%%%%%%%%%%%%%%%%%%%%%%%%%%%%%%%%
\subsubsection{ Pseudoprimes and Carmichael Numbers }
There are composite numbers that will behave similarly to primes when Fermat's theorem is applied to them for certain $a$. We will call these composites \textit{Fermat pseudoprimes}.
\begin{definition}
A \textit{ Fermat pseudoprime} base $a$ is an odd composite number $m$ for which 
$$ a^{m-1} \equiv 1 \pmod m$$  
holds. 
\end{definition}

 \par In general, a \textit{pseudoprime} is a composite number that behaves like a prime in the context of a primality or compositeness test. We will see other pseudoprimes in the future. If it were somehow possible to characterize all Fermat pseudoprimes for a base $a$, we could then use Fermat's Theorem with base $a$ as a primality test - this is approachable for fixed length numbers.
 
 \par The question arises - can all Fermat pseudoprimes be detected through a change of base? The answer, unfortunately, is no.   There exist numbers that will be Fermat pseudoprimes base $k$ for all valid $k$, and these numbers motivate the following definition.
 
 \begin{definition}
  A composite number $n$ such that $a^{n -1} \equiv 1 \pmod n$ for all $a$ relatively prime to $n$ is called a \textit{Carmichael Number}.
 \end{definition}
 
 A Carmichael number, the smallest of which is $561$, will never be revealed as composite by a Fermat's Theorem test. The existence of Carmichael numbers are a serious hindrance of the Fermat's theorem method, although fortunately they can be characterized.

\begin{theorem} $n$ is a Carmichael number if and only if $p-1$ divides $n-1$ for every prime factor of $n$, $n$ is composite, and $n$ is squarefree.
\end{theorem}

\begin{proof}
First, suppose that $n$ is a composite number that square free, and $p-1 \mid N-1$ for every prime factor $p$ of $n$, and that $a$ is an arbitrary integer such that $(a,n) = 1$.
\par $n$ is square free, so $n = p_1 p_2 p_3 \ldots p_k$. Because $a$ is relatively prime to $n$, $a$ is also relatively prime to $p_i$ for all $i \in {1, \ldots, k}$. Thus, $a^{p-1} \equiv 1 \pmod{p_i}$ for all $i$ by Fermat's Theorem. However, because $p -1 \mid n -1$, $a^{n-1} = (a^{p-1})^t \equiv (1)^t = 1 \pmod{p_i}$ for some $t \in \mathbb Z$. Because $p_i \mid a^{n-1} - 1$ for all $i$, and all $p_i$ are distinct primes, $n = p_1p_2p_3 \ldots p_k \mid a^{n-1} - 1$. Thus $a^{n-1} \equiv 1 \pmod N$ for any arbitrary $a$ relatively prime to $n$, and $n$ is a Carmichael number.
\par
We now suppose that a composite $n$ is a Carmichael number, so $a^{n-1} \equiv 1 \pmod n$ for all $a$ relatively prime to $n$. $n = p_1^{\alpha_1}\cdot \ldots p_k^{\alpha_k}$, and so $p_i^{\alpha_i} \mid a^{n-1} - 1 $ and $(a,p_i) = 1$ for all $i \in \{1, \ldots, k\}$. Let $r$ be an integer such that $N = p_i^{\alpha_i} r$. Obviously, $p_i^{\alpha_i}$ and $r$ are relatively prime, so by the chinese remainder theorem we have a unique solution of the system
	$$x \equiv a_1 \pmod{p_i^{\alpha_i}}$$
	$$x \equiv 1 \pmod{r}$$
where $a_1$ is a primitive root of $p_i^{\alpha_i}$. 
\par Suppose that $\alpha_i = 1$. $x$ is coprime to both $p_i$ and $r$, so we have $x \equiv a \pmod{p_i \cdot r = n}$, and so $x^{n-1} \equiv 1 \pmod n$, and so $x^{n-1} \equiv 1 \pmod p_i$ as well. However, $x^{p_i - 1} \equiv 1 \pmod{p_i}$, and $p_i - 1$ is the least power of $x$ such that $x$ is equivalent to $1$ modulo $p_i$, and so we have $p_i -1 \mid N - 1$ for all $i$. 
\par Now suppose that $\alpha_i >1$ for some $i$, or that $n$ is not squarefree. $x^{n-1} \equiv 1 \pmod{p_i^{\alpha_i}}$, and by Euler's theorem, $x^{\phi(p_i^{\alpha_i})} = x^{p_i^{\alpha_i - 1}(p_i - 1))}  \equiv 1 \pmod{p_i^{\alpha_i}}$. Thus because $p_i^{\alpha_i -1}(p_i - 1)$ is the least power of $a_1$ equivalent to $1$ modulo $pi^{\alpha_i}$, $p_i^{\alpha_i -1}(p_i - 1) \mid n - 1$. However, $p_i^{\alpha_i - 1} \mid n - 1$ is a contradiction, because $(n, n-1) = 1$ and $p_i^{\alpha_i - 1} \mid n$. Thus $n$ must also be squarefree, and we are done. 
\end{proof}


\subsubsection{Improvements due to Eulers Criterion}

There is clear motivation to improve our method, if possible, to avoid the problems presented by Carmichael numbers. One tool that we have at our disposal is the \textit{Euler Criterion for Quadratic Residues}, which is is stated below and proven in the Background %%ref 
section.

\newtheorem*{eulercriterion}{Euler's Criterion for Quadratic Residues}
\newtheorem*{eulerconverse}{Euler's Criterion as a Compositeness Test}

\begin{eulercriterion}
If $p$ is an odd prime and $(a,p) = 1$, 
$$a^{\frac{n-1}{2}} \equiv \pm 1 \pmod p$$
\end{eulercriterion}


This criterion's logical converse, much like Fermat's theorem, gives us a test for compositeness.
\begin{eulerconverse}
If $n$ is odd, $(a,n) = 1$, and
	$$a^{\frac{n-1}{2}} \not\equiv \pm 1 \pmod n$$
	then $n$ is a composite number.
\end{eulerconverse}

To strengthen this compositeness test, we use \textit{Jacobi's symbol}, $( \frac{a}{n})$, over Legendre's symbol.

\begin{eulerconverse}
If $n$ is odd, $(a,n) = 1$, and
	$$a^{\frac{n-1}{2}} \not\equiv \pm 1 \pmod n$$
	then $n$ is a composite number.
	
\par If $a^{\frac{n-1}{2}} \equiv \pm 1 \pmod n$ and
	$$a^{\frac{n-1}{2}} \not\equiv (\frac{a}{n}) \pmod n$$
	then $n$ is composite.
	
\par However, if $a^{\frac{n-1}{2}} \equiv (\frac{a}{n}) \pmod n$ holds, the test is inconclusive.
\end{eulerconverse}

As the last line of our definition implies, this test also has pseudoprimes associated with it. We will call these \textit{Euler pseudoprimes}.
\subsubsection{Euler Pseudoprimes and Strong Pseudoprimes}
We define an Euler pseudoprime in an analogous way to our definition of a Fermat pseudoprime.
\begin{definition}
Let $n$ be an odd composite number such that
	 $$a^{\frac{n-1}{2}} \not\equiv (\frac{a}{n}) \pmod n$$ 
for some $a$ relatively prime to $n$.  Then $n$ is an \textit{Euler pseudoprime} base $a$.
\end{definition}

Unfortunately, there are some numbers that are both Carmichael numbers and Euler pseudoprimes. An example of one such number is $1729$ - %% flesh out this example later
\par Despite this disappointing fact,  many Carmichael numbers are revealed as composite by Euler's criterion, and additionally, there is no analogue for Carmichael numbers in Euler pseudoprimes. If enough bases are tested, we will eventually prove an Euler pseudoprime composite.

\par To close this section, we present a final type of pseudoprime.
\begin{definition}
An odd composite number $n$ with $n-1 = d\cdot2^s$, $d$ odd, is called a \textit{strong pseudoprime} for base $a$ if either
$a^d \equiv 1 \pmod n$ or $a^{d \cdot 2r} \equiv -1 \pmod n$, for some $r \in \{0, 1, \ldots, s-1\}$. 
\end{definition}

\par This test for pseudoprimes is intended, much like Euler pseudoprimes, to eliminate the problem of Carmichael numbers in the Fermat
Compositeness test. Indeed, any Fermat pseudoprime will be eventually proven composite by the strong pseudoprime test if enough 
bases $a$ are tested.

\par The motivation for the definition is as follows - Any Fermat pseudoprime $n$ to base $a$ will satisfy the equivalence
	$$a^{n-1} - 1 \equiv 0 \pmod n$$
Because we assume that $n$ is odd if it is being tested for primality, $n = 2m+1$ for some integer $m$ and we have
	$$a^{2m} - 1 = (a^m - 1)(a^m+1) \equiv 0 \pmod(n)$$
If $n$ is a prime, it must divide one of these factors, but it cannot divide both because then it would need to divide all linear
combinations of them, including $(a^m - 1) - (a^m + 1) = -2$. Therefore we have 
	$$a^m \equiv \pm 1 \pmod n$$
We can write $n$ as $2^\alpha k + 1$, where $k$ is odd, and have 
	$$a^{n-1} - 1 = (a^k - 1)(a^k + 1)(a^{2k} + 1) \ldots (a^{2^{\alpha - 1} k} +1)$$
	
\par If $n$ divides exactly one of these factors but is composite, then it is a strong pseudoprime. Interestingly, if a number is
a strong pseudoprime to the base $a$, it will also be a Euler pseudoprime to $A$.





%%%%%%%%%%%%%%%%%%%%%%%%%%%%%%%%%%%%%%%%%%%%%%%%%%%%%%%
%%%%%%%%%%%%%%%%%%%%%%%%%%%%%%%%%%%%%%%%%%%%%%%%%%%%%%%
%%%%%%%%%%%%%%%%%%%%%%%%%%%%%%%%%%%%%%%%%%%%%%%%%%%%%%%
%%%%%%%%%%%%%%%%%%%%%%%%%%%%%%%%%%%%%%%%%%%%%%%%%%%%%%%



\subsection{Proving Primality}

\subsubsection{Lehmer's Theorem}

\subsubsection{Selfridge's Theorem}

\subsubsection{Pepin's Theorem}

\subsubsection{A Relaxed Primality Test}

\subsection{Identifying Primes of a Certain Form}

\subsubsection{Proth's Theorem}

\subsubsection{Lucas Sequences}

\subsubsection{Pocklington's Theorem}

\subsubsection{Primes of other forms}

%%%%%%%%%%%%%%%%%%%%%%%%%%%%%%%%%%%%%%%%%%%%%%%%%%%%%%%
%%%%%%%%%%%%%%%%%%%%%%%%%%%%%%%%%%%%%%%%%%%%%%%%%%%%%%%
%%%%%%%%%%%%%%%%%%%%%%%%%%%%%%%%%%%%%%%%%%%%%%%%%%%%%%%
%%%%%%%%%%%%%%%%%%%%%%%%%%%%%%%%%%%%%%%%%%%%%%%%%%%%%%%

\subsection{Modern Primality Tests}

%% We will treat these very lightly 

\subsubsection{The Jacobi Sum Primality Test}

\subsubsection{Lenstra's Theorem}

\subsubsection{Elliptic Curve Primality Testing}

\subsubsection{The Goldwasser-Kilian Test}

\subsubsection{Atkin's Test}

%%%%%%%%%%%%%%%%%%%%%%%%%%%%%%%%%%%%%%%%%%%%%%%%%%%%%%%
%%%%%%%%%%%%%%%%%%%%%%%%%%%%%%%%%%%%%%%%%%%%%%%%%%%%%%%
%%%%%%%%%%%%%%%%%%%%%%%%%%%%%%%%%%%%%%%%%%%%%%%%%%%%%%%
%%%%%%%%%%%%%%%%%%%%%%%%%%%%%%%%%%%%%%%%%%%%%%%%%%%%%%%

\subsection{Closing Notes on Primality Testing}
\end{document}
