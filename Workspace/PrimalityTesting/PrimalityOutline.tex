%%%%%%%%%%%%%%%%%%%%%%%%%%%%%%%%%%%%%%%%%%%%%%%%%%%%%%%
% % 
% % 	Introduction Draft
%%
% %	 Math 371: 		Number Theory, Spring 2016
% % 	Author(s): 		Margaret Dorsey
% % 	Last Updated: 	04 / 16 / 2016	(see gitlog.txt for revision history)
% % 
%%%%%%%%%%%%%%%%%%%%%%%%%%%%%%%%%%%%%%%%%%%%%%%%%%%%%%%
% % Title and author(s)
%%%%%%%%%%%%%%%%%%%%%%%%%%%%%%%%%%%%%%%%%%%%%%%%%%%%%%%


\documentclass{article}
\usepackage{graphicx}
\newenvironment{intro}
  {\renewcommand\abstractname{Introduction}\begin{abstract}}
  {\end{abstract}}



%%%%%%%%%%%%%%%%%%%%%%%%%%%%%%%%%%%%%%%%%%%%%%%%%%%%%%%
% % There are a lot of commands for theorems, declarations, etc.
% %
% % Here are some common ones. 
%%		-Margaret
%%%%%%%%%%%%%%%%%%%%%%%%%%%%%%%%%%%%%%%%%%%%%%%%%%%%%%%
%%
%%\newtheorem{theorem}{Theorem}
%%\newtheorem{lemma}{Lemma}
%%\newtheorem{proposition}{Proposition}
%%\newtheorem{scolium}{Scolium}   %% Maybe common was an overstatement..
%%\newtheorem{definition}{Definition}
%%\newenvironment{proof}{{\sc Proof:}}{~\hfill QED}
%%\newenvironment{AMS}{}{}
%%\newenvironment{keywords}{}{}
%%
%%%%%%%%%%%%%%%%%%%%%%%%%%%%%%%%%%%%%%%%%%%%%%%%%%%%%%%

\begin{document}
\newpage


%%%%%%%%%%%%%%%%%%%%%%%%
%%
%%  \section{ sectionName }
%%	\subsection{ name}
%%	\subsubsection{name}
%%	etc..
%%  Use sectioning commands for headings. Often longer articles are divided into a few sections.
%%
%%%%%%%%%%%%%%%%%%%%%%%%%%%%%%%%
%%
%%  use /par to start a new paragraph, // for a simple line break.
%%
%%  You must use labelling commands, e.g. \label, \ref,
%%  \bibitem, and \cite to refer to sections of your document, such as
%%  see Section~\ref{sectionname}, see Figure~\ref{figurename}, or bibliography entries,
%%  such as see~\cite{citationname}.  Otherwise the look of the numbers, and sometimes the numbers
%%  themselves, will get messed up.
%%
%%%%%%%%%%%%%%%%%%%%%%%%%%%%%%%%%

\section{Primality Testing}

% Introduction to the subject

%%%%%%%%%%%%%%%%%%%%%%%%%%%%%%%%%%%
\subsection{ The Sieve of Eratosthenes }


%%%%%%%%%%%%%%%%%%%%%%%%%%%%%%%%%%%
\subsection{ Primality Tests and Composite Tests}

%%%%%%%%%%%%%%%%%%%%%%%%%%%%%%%%%%%
\subsection {Fermat's Theorem and Resulting Methods}
	
\subsubsection{Fermat's Theorem}

\subsubsection{Pseudoprimes and Carmichael Numbers}

%%%%%%%
%%
%% A possible proof: N is a Carmichael number if and only if p-1 divides N-1 for every prime factor of N, N is composite,
%% and N is square free.
%%
%%%%%%%

\subsubsection{Computational Viability of Fermat's Theorem Methods}

\subsubsection{Improvements due to Eulers Criterion}

\subsubsection{Euler Pseudoprimes}

\subsubsection{Strong Pseudoprimes}

\subsubsection{Implementation of a Strong Pseudoprime Test}

\subsubsection{Remarks and Numerical Data}

%%%%%%%%%%%%%%%%%%%%%%%%%%%%%%%%%%%


\end{document}
