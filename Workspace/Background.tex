%%%%%%%%%%%%%%%%%%%%%%%%%%%%%%%%%%%%%%%%%%%%%%%%%%%%%%%
% % 
% % 	Background 
%%
% %	 Math 371: 		Number Theory, Spring 2016
% % 	Author(s): 		Margaret Dorsey
% % 	Last Updated: 	04 / 19 / 2016	(see gitlog.txt for revision history)
% % 
%%%%%%%%%%%%%%%%%%%%%%%%%%%%%%%%%%%%%%%%%%%%%%%%%%%%%%%
% % Title and author(s)
%%%%%%%%%%%%%%%%%%%%%%%%%%%%%%%%%%%%%%%%%%%%%%%%%%%%%%%


\documentclass{article}
\usepackage{graphicx}
\usepackage{amsthm}
\usepackage{amsfonts}
\usepackage{amssymb}
\usepackage{amsmath}
\theoremstyle{plain}
\newenvironment{intro}
  {\renewcommand\abstractname{Introduction}\begin{abstract}}
  {\end{abstract}}



%%%%%%%%%%%%%%%%%%%%%%%%%%%%%%%%%%%%%%%%%%%%%%%%%%%%%%%
% % There are a lot of commands for theorems, declarations, etc.
% %
% % Here are some common ones. 
%%		-Margaret
%%%%%%%%%%%%%%%%%%%%%%%%%%%%%%%%%%%%%%%%%%%%%%%%%%%%%%%
%%
%%\newtheorem{theorem}{Theorem}
%%\newtheorem{lemma}{Lemma}
%%\newtheorem{proposition}{Proposition}
%%\newtheorem{scolium}{Scolium}   %% Maybe common was an overstatement..
%%\newtheorem{definition}{Definition}
%%\newenvironment{proof}{{\sc Proof:}}{~\hfill QED}
%%\newenvironment{AMS}{}{}
%%\newenvironment{keywords}{}{}
%%
%%%%%%%%%%%%%%%%%%%%%%%%%%%%%%%%%%%%%%%%%%%%%%%%%%%%%%%

\begin{document}
\newpage



%%%%%%%%%%%%%%%%%%%%%%%%
%%
%%  \section{ sectionName }
%%	\subsection{ name}
%%	\subsubsection{name}
%%	etc..
%%  Use sectioning commands for headings. Often longer articles are divided into a few sections.
%%
%%%%%%%%%%%%%%%%%%%%%%%%%%%%%%%%
%%
%%  use /par to start a new paragraph, // for a simple line break.
%%
%%  You must use labelling commands, e.g. \label, \ref,
%%  \bibitem, and \cite to refer to sections of your document, such as
%%  see Section~\ref{sectionname}, see Figure~\ref{figurename}, or bibliography entries,
%%  such as see~\cite{citationname}.  Otherwise the look of the numbers, and sometimes the numbers
%%  themselves, will get messed up.
%%
%%%%%%%%%%%%%%%%%%%%%%%%%%%%%%%%%

\section {Background}


In this section, we examine a few topics that will be necessary in the main paper of which the reader may be unaware. 
%% This is by no means a definitive list or even
%% an indication that these will end up being covered here and not in the actual paper proper
%% I just wanted to keep track of knowledge that will be neccessary for the reader, and this seemed appropriate
%% - Margaret
\newtheorem*{definition}{Definition}
\subsection{Primitive Roots}
\begin{definition}
A number $g$ is a \textit{primitive root} modulo $m$ if every number $a$ coprime to $n$ is congruent to some power of $g$ modulo $n$. This $k$ is called the index of $a$ to the base $g$ modulo $n$.
\end{definition}
\par The primitive residue classes modulo $n$ have many useful properties, but we will not examine them very thoroughly - we only state without proof that $n$ has a primitive root if it is of the form $2, 4, p^k, \text{ or } 2p^k$ where $p$ is an odd prime and $k \geq 1$, and if an integer $n$ has a primitive root, then it has $\phi(\phi(n))$ of them. Additionally, the lowest power of a primitive root $a$ modulo $n$ that is equivalent to $1$ modulo $n$ is $\phi(n)$.

\subsection{Quadratic Residues}
\newtheorem*{quadraticresiduedefinition}{Quadratic Residue}
\begin{quadraticresiduedefinition}
Let $a$ and $n$ be integers such that $(a,n) = 1$. If the congruence
	$$x^2 \equiv a \pmod n$$
has solutions $x$, then $a$ is a \textit{quadratic residue} of $n$. If there are no such solutions, then $a$ is a \textit{quadratic non-residue} modulo $n$.
\end{quadraticresiduedefinition}

\par It should be noted that the condition $(a,n) = 1$ allows us to consider only the so-called \textit{primitive} residue classes modulo $n$, or those classes that are relatively prime to $n$ when searching for quadratic residues. 

\par We now have a result related to quadratic residues.
 \newtheorem*{theorem}{Theorem}
 \begin{theorem}
 The product of two quadratic residues $a$ and $b$ modulo $n$ is always a quadratic residue of $n$, and the product of two quadratic non-residues $\alpha$ and $\beta$ modulo $n$ is always a quadratic non-residue.
 \end{theorem}
 \begin{proof}
 Suppose that two integers $a$ and $a$ both relatively prime to $n$ are quadratic residues modulo $n$. Thus $x^2 \equiv a
  \pmod n$ and $y^2 \equiv b \pmod n$ for some $x$ and $y$. Because $a$ and $b$ are relatively prime to $n$, we can say
   $x^2y^2 = (xy)^2 \equiv ab \pmod n$ , which gives us our first result.
 \par Now suppose there is an $x$ such that  $x^2 \equiv \alpha \pmod n$, but no $y$ such that  $y^2 \equiv \beta \pmod n$,
  and there exists some $z$ such that $z^2 \equiv \alpha\beta \pmod n$. This gives us $z^2 \equiv x^2 \beta \pmod n$ , and thus 
  $\left( \frac{z}{x} \right)^2 \equiv \beta \pmod n$, which contradicts our assumption that $\beta$ is a non-quadratic residue.
 \end{proof}
 
 \par Note that we do not address the product of two non-residues - we will state without proof that for primes, the product of two non-residues is a residue, and that the issue is more complex for composite numbers. 
 
 
\subsubsection{Legendre's Symbol, Euler's Criterion, and Jacobi's Symbol}
We define Legendre's Symbol $\left( \frac{a}{p} \right)$ as a symbol given the value $1$ if $a$ is a quadratic residue of $p$ and the value $-1$ if $a$ is a quadratic non-residue of $p$, where $p$ is a prime.
\par It is, of course, possible to compute Legendre's Symbol directly by trying every congruence class of $p$, but luckily there is a more elegant way, given by Euler:

\newtheorem*{eulercriterion}{Euler's Criterion}
\begin{eulercriterion}
If $(a,p)  = 1$ and $p$ is an odd prime, then 
	$$ \left( \frac{a}{p} \right)  \equiv a^{\frac{p-1}{2}} \pmod p$$
\end{eulercriterion}
\begin{proof}
The result can be broken into two results:
	$$a^{\frac{p-1}{2}} \equiv 1 \pmod p \text{ if and only if } a \text{ is a quadratic residue of } p$$
	$$\text{ and }$$
	$$a^{\frac{p-1}{2}} \equiv -1 \pmod p \text{ if and only if } a \text{ is a quadratic non-residue of } p$$
	
\par We begin with the second of these. Let $a$ be a quadratic non-residue of a prime $p$, and $b$ be some natural number less than $p$. $bx \equiv a$ has a unique solution $x = b^{-1}$, because $(b,p) = 1$. However, $b \not \equiv b^{-1} \pmod p$, because otherwise $b^2 \equiv a \pmod p$ contradicts the assumption that $a$ is a quadratic non-residue. Thus all natural numbers less than $p$ can be paired into $\frac{p-1}{2}$ pairs $(m,n)$ such that $mn \equiv a \pmod p$. 
\par Multiplying these pairs together, we have a product of  $\frac{p-1}{2}$ integers all equivalent to $a$ modulo $p$, which will be equivalent to $a^\frac{p-1}{2}$ modulo $p$ and  equal to $(p-1)!$. By Wilson's Theorem, we also have $(p-1)! \equiv -1 \pmod p$, and so when $a$ is a quadratic non-residue of $p$, $a^\frac{p-1}{2} \equiv -1 \pmod p$.
\par Now let $a$ be such that $a^\frac{p-1}{2} \equiv -1 \pmod p$. By Wilson's Theorem, $a^\frac{p-1}{2} \equiv (p-1)! \pmod p$, and thus this implies that the product of all natural  numbers $b < p$ produce $\frac{p-1}{2}$ factors equivalent to $a$. If any $b$ exists such that $bx \equiv a \pmod p$ has its unique solution $x$ equivalent to $b$, then $b$ will not be able to form a product equivalent to $a$ when paired with any other factor of $(p-1)!$, which is a contradiction. Thus there is no $b$ such that $b^2 \equiv a \pmod p$, and $a$ is a quadratic non-residue.
\par Now let $a$ be a quadratic residue of $p$. We can pick a natural number $b < p$ such that $b^2 \equiv a \pmod p$, and thus $ b^{p-1} \equiv a^\frac{p-1}{2} \pmod p$. By Fermat's Theorem, we have $ a^\frac{p-1}{2} \equiv 1 \pmod p$.
\par Finally, we examine the case that  $a^\frac{p-1}{2} \equiv 1 \pmod p$ for some arbitrary $a$ relatively prime to $p$. Let $b$ be a primitive root modulo $p$ such that $a$ can be written as $b^j$ for some $j$. From this we have $b^{j \cdot \frac{p-1}{2}} \equiv 1 \pmod p$. Because the least power of $b$ that is equivalent to $1$ modulo $p$ is $p-1$, $p -1 \mid j \cdot \frac{p-1}{2}$. Thus $j$ must be even, and $\frac{j}{2}$ is an integer. Thus we have that $(b^{\frac{j}{2}})^2 \equiv a \pmod p$, and thus $a$ must be a quadratic residue.
	
\end{proof}

\par You may have noticed that the Legendre Symbol is only defined for a prime $p$. We define an analogue for composites, \textit{Jacobi's Symbol}, $\left( \frac{a}{n} \right)$, as $\prod_i \left( \frac{a}{p_i} \right) ^{\alpha_i}$ for $n$ an odd integer, $(a,n) = 1$, and $n = \prod_i p_i^{\alpha_i}$. 

\par Recalling that we never addressed the case of the product of two quadratic non residues of a composite number, it makes intuitive sense that the Jacobi symbol occasionally incorrectly takes the value $1$ for a quadratic non-residue, given it is the product of two non-residues. However, it never incorrectly identifies a quadratic residue.  Thus it can be used to quickly determine that something is not a quadratic residue modulo $n$, but cannot prove something as a quadratic residue mod $n$.

%\subsection{Modules}

%\begin{definition}
%A set of numbers $M$ is called a module if for every $x$ and $y$ in $M$ $x+y$ and $x-y$ are also in $M$.
%\end{definition}
%We can view modules as sets closed under subtraction, because if $x-y \in M$, $x - x = 0 \in M$, and consequently $0 - y = - y \in M$ so $x - (-y) = x +y \in M$. Examples of modules include the set of integers, and the set of all integer multiples of a real number $\alpha$.

%\par We have the following theorem for a module $M \subseteq \mathbb Z$.
%\begin{theorem}
%All elements of a module $M \subseteq \mathbb Z$, $M$ not equal to $\{0\}$, are multiples of an integer $d$, the smallest positive %integer contained in $M$.
%\end{theorem}
%\begin{proof}
%Let $d$ be the smallest positive element of $M$. All multiples of $d$ must be in $M$ as well, because we can continually sum $d$ to %obtain positive multiples, and $-d$ to obtain negative multiples. Also, $d - d = 0 \in M$. Suppose that there is some integer $x$ in %$M$ that is not a multiple of $d$. Thus $ nd < x < (n+1)d$ for some $n \in \mathbb Z$. But then $x - nd \in M$ as well, and that %element is less than $d$, which condradicts the minimality of $d$. So there is no integer in $M$ that is not a multiple of the least %positive integer $d$.
%\end{proof}

\par It should be noted that the smallest module $M$ containing two integers $a$ and $b$ is the one generated by $d = (a,b)$. The proof is trivial.

\subsection{A Few Small Results}
\begin{theorem}
If $k$ is the least positive integer such that $a^k \equiv 1 \pmod n$, $a^j \equiv 1 \pmod n$, and $k < j$, then $k \mid j$.
\end{theorem}
\begin{proof}
Let $k$ be the least power such that $a^k \equiv 1 \pmod n$. $j = k + m$, with $m \in \mathbb N$. if $k \nmid m$, then $m = qk+ r$, $ 0 < r < m$, and $j = (q+1)k + r$. However, $a^j = a^{(q+1)k}a^r \equiv a^r \equiv 1 \pmod n$ contradicts the minimality of $k$. Thus $k$ must divide $j$.
\end{proof}

\end{document}
