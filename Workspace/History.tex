%%%%%%%%%%%%%%%%%%%%%%%%%%%%%%%%%%%%%%%%%%%%%%%%%%%%%%%
% % 
% % 	History Draft
%%
% %	 Math 371: 		Number Theory, Spring 2016
% % 	Author(s): 		Jodie Miu
% % 	Last Updated: 	04 / 23 / 2016	(see gitlog.txt for revision history)
% % 
%%%%%%%%%%%%%%%%%%%%%%%%%%%%%%%%%%%%%%%%%%%%%%%%%%%%%%%
% % Title and author(s)
%%%%%%%%%%%%%%%%%%%%%%%%%%%%%%%%%%%%%%%%%%%%%%%%%%%%%%%


\documentclass{article}
\usepackage{graphicx}
\usepackage{amsmath}



%%%%%%%%%%%%%%%%%%%%%%%%%%%%%%%%%%%%%%%%%%%%%%%%%%%%%%%
% % There are a lot of commands for theorems, declarations, etc.
% %
% % Here are some common ones. 
%%		-Margaret
%%%%%%%%%%%%%%%%%%%%%%%%%%%%%%%%%%%%%%%%%%%%%%%%%%%%%%%
%%
%%\newtheorem{theorem}{Theorem}
%%\newtheorem{lemma}{Lemma}
%%\newtheorem{proposition}{Proposition}
%%\newtheorem{scolium}{Scolium}   %% Maybe common was an overstatement..
%%\newtheorem{definition}{Definition}
%%\newenvironment{proof}{{\sc Proof:}}{~\hfill QED}
%%\newenvironment{AMS}{}{}
%%\newenvironment{keywords}{}{}
%%
%%%%%%%%%%%%%%%%%%%%%%%%%%%%%%%%%%%%%%%%%%%%%%%%%%%%%%%

\begin{document}
\newpage



%%%%%%%%%%%%%%%%%%%%%%%%
%%
%%  \section{ sectionName }
%%	\subsection{ name}
%%	\subsubsection{name}
%%	etc..
%%  Use sectioning commands for headings. Often longer articles are divided into a few sections.
%%
%%%%%%%%%%%%%%%%%%%%%%%%%%%%%%%%
%%
%%  use /par to start a new paragraph, // for a simple line break.
%%
%%  You must use labelling commands, e.g. \label, \ref,
%%  \bibitem, and \cite to refer to sections of your document, such as
%%  see Section~\ref{sectionname}, see Figure~\ref{figurename}, or bibliography entries,
%%  such as see~\cite{citationname}.  Otherwise the look of the numbers, and sometimes the numbers
%%  themselves, will get messed up.
%%
%%%%%%%%%%%%%%%%%%%%%%%%%%%%%%%%%


\par Although the computational necessity for factorization and primality testing methods is patently obvious to us in the modern 
information age, the study and classification of primes and interest in the factorization of large numbers dates back to at least the Ancient Greeks. The notion of primes is at least as old as Euclid (circa 300 BCE), considering a definition of primality is contained in his \textit{ Elements} 
%\citep{historyref}
, and Eratosthenes' famous sieve, the earliest known method for algorithmic identification of primes and factorization of composites, was soon to follow in the $3^{\text{rd}}$ century BCE. 
\par It is a fairly safe assumption that these first forays into primality testing and factorization were more motivated by curiosity than necessity -  and indeed, it was not until the modern era that truly efficient and sophisticated methods were developed - but with the preservation of Greek mathematics by Arabic scholars, the theory continued to develop until the first deterministic primality test by trial division was outlined by Fibonacci in the early $13^{\text{th}}$ century. Further developments in the field of primality and factorization came with the study of perfect numbers by the Italian mathematician Cataldi, whose observations on the relationship between perfect numbers and primality were later made rigorous by Fermat, and the French priest Mersenne, who developed his famous Mersenne primes.
\par In the $17^{\text{th}}$ century came Fermat and his little theorem, which as we will see, forms the basis for a vast number of primality tests. This result, and more similar to it, arose from his study of Mersenne primes and their classification, which then developed into the discovery of so called Fermat numbers, the only numbers of the form $2^n + 1$ that can be prime. Fermat was also credited for the development of the difference of squares method of factoring, a much more sophisticated method than trial division or Eratosthenes' sieve.
\par With the $18^{\text{th}}$ century came Euler, and a veritable landslide of significant Number Theoretic results. Continuing from the work of Fermat, Euler provided many results, some covered in this paper, that further refined the theory of primality, factorization, and classification of primes. Soon after came Legendre, whose work in quadratic residues (along with the work of Euler and Gauss in the same field) allowed him to develop factorization methods based on quadratic sieves, which inform some of the methods still seen today.
\par Gauss, who had a deep interest in primality testing and factorization methods, also made significant strides in the field, both extending Legendre's sieve and contributing many results and methods of his own. After Gauss came the nineteenth and twentieth centuries and the work of Reuschle, Landry, Lucas, and Lehmer, all of whom significantly contributed to the more contemporary methods for factorization and primality testing, and with the advent of the computer age, the development of truly modern methodologies was not far behind. 
\par It is unlikely that many of the mathematicians that built the foundations of primality testing and factorization methods had any idea that it would become such an important area of study in its own right, but the fact that their study was motivated by other problems and pursuits long before use in cryptographic systems and big data remains. Although many of the results discussed in this paper are stated without context, it is useful to remember that they were generally developed for a wide variety of purposes, very few of which align with their current applications, and that the sophisticated methods of today developed only after centuries of refinements made to the crude elementary techniques of trial division and straightforward sieves.

 \end{document}
