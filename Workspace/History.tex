%%%%%%%%%%%%%%%%%%%%%%%%%%%%%%%%%%%%%%%%%%%%%%%%%%%%%%%
% % 
% % 	History Draft
%%
% %	 Math 371: 		Number Theory, Spring 2016
% % 	Author(s): 		Jodie Miu
% % 	Last Updated: 	04 / 23 / 2016	(see gitlog.txt for revision history)
% % 
%%%%%%%%%%%%%%%%%%%%%%%%%%%%%%%%%%%%%%%%%%%%%%%%%%%%%%%
% % Title and author(s)
%%%%%%%%%%%%%%%%%%%%%%%%%%%%%%%%%%%%%%%%%%%%%%%%%%%%%%%


\documentclass{article}
\usepackage{graphicx}
\newenvironment{history}
  {\renewcommand\abstractname{History}\begin{abstract}}
  {\end{abstract}}



%%%%%%%%%%%%%%%%%%%%%%%%%%%%%%%%%%%%%%%%%%%%%%%%%%%%%%%
% % There are a lot of commands for theorems, declarations, etc.
% %
% % Here are some common ones. 
%%		-Margaret
%%%%%%%%%%%%%%%%%%%%%%%%%%%%%%%%%%%%%%%%%%%%%%%%%%%%%%%
%%
%%\newtheorem{theorem}{Theorem}
%%\newtheorem{lemma}{Lemma}
%%\newtheorem{proposition}{Proposition}
%%\newtheorem{scolium}{Scolium}   %% Maybe common was an overstatement..
%%\newtheorem{definition}{Definition}
%%\newenvironment{proof}{{\sc Proof:}}{~\hfill QED}
%%\newenvironment{AMS}{}{}
%%\newenvironment{keywords}{}{}
%%
%%%%%%%%%%%%%%%%%%%%%%%%%%%%%%%%%%%%%%%%%%%%%%%%%%%%%%%

\begin{document}
\newpage



%%%%%%%%%%%%%%%%%%%%%%%%
%%
%%  \section{ sectionName }
%%	\subsection{ name}
%%	\subsubsection{name}
%%	etc..
%%  Use sectioning commands for headings. Often longer articles are divided into a few sections.
%%
%%%%%%%%%%%%%%%%%%%%%%%%%%%%%%%%
%%
%%  use /par to start a new paragraph, // for a simple line break.
%%
%%  You must use labelling commands, e.g. \label, \ref,
%%  \bibitem, and \cite to refer to sections of your document, such as
%%  see Section~\ref{sectionname}, see Figure~\ref{figurename}, or bibliography entries,
%%  such as see~\cite{citationname}.  Otherwise the look of the numbers, and sometimes the numbers
%%  themselves, will get messed up.
%%
%%%%%%%%%%%%%%%%%%%%%%%%%%%%%%%%%

\begin{History}
\par The concept of primality probably originated with the ancient Greeks over $2.5$ millennia ago.
In fact, the very first recorded definition of prime numbers came from Euclid's \textit{Elements}
around 300 BCE. Interestingly enough, there is indirect evidence that the concept of primality
may have been known earlier to Pythagoras and his followers.

\par The Greeks called what we know today as number theory arithmetic. The first sieve was proposed
by Erasthosthenes (284-204 BCE). The Sieve of Erasthosthenes is the only known algorithm from
antiquity for primes that can be called a primality test. Today, however, it is considered a highly
inefficient test.

\par The Arabs preserved much mathematics from antiquity. Ibn al-Banna (ca 1258-1334) appears to be
the first to observe that when using the Sieve of Erasthosthenes, one can restrict attention to
prime divisors less than $\sqrt{n}$.

\par Leonardo of Pisa, more famously known as Fibonacci (ca 1170-1250), is a good example of the resurrection of
mathematical interest in Europe during the 13th century. He was tutored by an Arab scholar while living in
North Africa, and later published \textit{Liber Abaci}, or the \textit{Book of Calculation}. In his book,
Fibonacci gave an algorithm to determine if $n$ is prime  by dividng $n$ by natural numbers up to $\sqrt{n}$.
This is the first recorded instance of a deterministic algorithm, where a deterministic algorithm follows
the same sequence of operations when executed with the same input.
\end{History}

\end{document}
