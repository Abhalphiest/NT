%%%%%%%%%%%%%%%%%%%%%%%%%%%%%%%%%%%%%%%%%%%%%%%%%%%%%%%
% % 
% % 	Introduction Draft
%%
% %	 Math 371: 		Number Theory, Spring 2016
% % 	Author(s): 		Margaret Dorsey
% % 	Last Updated: 	04 / 16 / 2016	(see gitlog.txt for revision history)
% % 
%%%%%%%%%%%%%%%%%%%%%%%%%%%%%%%%%%%%%%%%%%%%%%%%%%%%%%%
% % Title and author(s)
%%%%%%%%%%%%%%%%%%%%%%%%%%%%%%%%%%%%%%%%%%%%%%%%%%%%%%%


\documentclass{article}
\usepackage{graphicx}
\newenvironment{intro}
  {\renewcommand\abstractname{Introduction}\begin{abstract}}
  {\end{abstract}}



%%%%%%%%%%%%%%%%%%%%%%%%%%%%%%%%%%%%%%%%%%%%%%%%%%%%%%%
% % There are a lot of commands for theorems, declarations, etc.
% %
% % Here are some common ones. 
%%		-Margaret
%%%%%%%%%%%%%%%%%%%%%%%%%%%%%%%%%%%%%%%%%%%%%%%%%%%%%%%
%%
%%\newtheorem{theorem}{Theorem}
%%\newtheorem{lemma}{Lemma}
%%\newtheorem{proposition}{Proposition}
%%\newtheorem{scolium}{Scolium}   %% Maybe common was an overstatement..
%%\newtheorem{definition}{Definition}
%%\newenvironment{proof}{{\sc Proof:}}{~\hfill QED}
%%\newenvironment{AMS}{}{}
%%\newenvironment{keywords}{}{}
%%
%%%%%%%%%%%%%%%%%%%%%%%%%%%%%%%%%%%%%%%%%%%%%%%%%%%%%%%

\begin{document}
\newpage


%%%%%%%%%%%%%%%%%%%%%%%%
%%
%%  \section{ sectionName }
%%	\subsection{ name}
%%	\subsubsection{name}
%%	etc..
%%  Use sectioning commands for headings. Often longer articles are divided into a few sections.
%%
%%%%%%%%%%%%%%%%%%%%%%%%%%%%%%%%
%%
%%  use /par to start a new paragraph, // for a simple line break.
%%
%%  You must use labelling commands, e.g. \label, \ref,
%%  \bibitem, and \cite to refer to sections of your document, such as
%%  see Section~\ref{sectionname}, see Figure~\ref{figurename}, or bibliography entries,
%%  such as see~\cite{citationname}.  Otherwise the look of the numbers, and sometimes the numbers
%%  themselves, will get messed up.
%%
%%%%%%%%%%%%%%%%%%%%%%%%%%%%%%%%%

\section{Prime Factorization}

% Introduction to the subject
\par Prime factorization is the method by which we factor an integer into its canonical prime factorization.
For relatively small integers, this is trivial to do by hand. As these integers scale, however, this task
becomes prohibitively expensive to compute by our modern day algorithms. In fact, the security of the RSA
cryptographic algorithm relies on the difficulty of factoring large algorithms.

\par Because factoring large integers is so computationally expensive, it is better to first determine that
the number we want to factor is, indeed, a composite. To do this, we can use the variety of primality tests
to ensure that our number is factorable before we start down this path.

%%%%%%%%%%%%%%%%%%%%%%%%%%%%%%%%%%%
\section{Classical Factorization Methods}

\subsection{ Trial Division }
\par As a factorization method, trial division is repeated division of our number $N$ by
small primes. We store the pre-computed primes and their number in some table. This would be very
speedy, albeit at the cost of the storage. Alternatively, we could save space by generating our 
primes along the way, using the form $6k \pm 1$ (including $2$ and $3$).

%%%%%%%%%%%%%%%%%%%%%%%%%%%%%%%%%%%
\subsection{ Euclid's Algorithm }
\par Interestingly enough, we can also use Euclid's algorithm to search for the factors of a
number. Euclid's algorithm will give us the prime factors of our number $N$ between $g$ and $G$, where
....

\par To use Euclid's algorithm to factor a number, start by multiplying together all primes between
the two limits. In fact, there even exists precomputed products for primes within certain ranges, to
use here should we want to. Next, we apply Euclid's algorithm on the product of the primes and our number
$N$.

\par The application of Euclid's is fast because it simply is repeated division, multiplication, and
subtraction. The only possible tricky thing here is the initial division of our long number by $N$, but
aside from that, it is not too difficult to perform in step-wise fashion.

\par The use of Euclid's algorithm to factor numbers is not obsolete. In fact, it is used when $N$ is
too large to be easily divided by a small prime. It is also used when only small divisors are sought.


%%%%%%%%%%%%%%%%%%%%%%%%%%%%%%%%%%%
\subsection{ Fermat }
\par The idea behind Fermat's method of factoring is that we can write an odd composite number as a difference
between two squares. Once we have that, it naturally gives us a factorization, in that: $N = x^2 - y^2 = (x-y)(x+y)$.

\par To find the two square numbers, we start by making the observation that any $x$ that satisfies $N=x^2-y^2$ 
must be $> \sqrt{N}$. Thus, we start with $m=\floor{\sqrt{N}}+1$, which is the smallest $x$ possible (aside
from the case where $N$ is $x^2$).

\par We consider $x=m^2-N$, checking to see if it is a square. If it is, then we have found our $x$ and $y$.

%%%%%%%%%%%%%%%%%%%%%%%%%%%%%%%%%%%
\subsection{A Revisit of The Sieve of Eratosthenes}

%%%%%%%%%%%%%%%%%%%%%%%%%%%%%%%%%%%
\subsubsection {Construction of a Factor Table}
	


%%%%%%%%%%%%%%%%%%%%%%%%%%%%%%%%%%%


\end{document}
